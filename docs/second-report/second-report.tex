\documentclass[10pt,a4paper]{article}

\usepackage[margin=1.5cm]{geometry}
\usepackage[UKenglish]{babel}
\usepackage{enumitem}
\usepackage{fancyhdr}
\usepackage{graphicx}
\usepackage{multirow}
\usepackage[table]{xcolor}
\usepackage{float}
\usepackage{longtable}
\usepackage{parskip}
\usepackage[small,compact]{titlesec} 

\definecolor{titleColor}{RGB}{138,201,242}
\pagestyle{fancy}
\lhead{T Davies, A Fahie, A Fairbairn, A Free, J Mansfield, R Tucker, M Walker}
\chead{}
\rhead{GPIG-C}
\cfoot{\vspace{-0.6cm} \thepage}

\setlist{nolistsep} % Reduces lots of white space around lists

\renewcommand{\headrulewidth}{0.4pt} % Add rules below header
\renewcommand*{\thefootnote}{\fnsymbol{footnote}}

\begin{document}

\begin{center}
{\Large GPIG-C Interim Report}

Word count: @WORD_COUNT@
\unskip
\footnote{\textit{Using TeXCount, excluding\ldots}}

Friday, 14th February 2014
\end{center}

\vspace{0.3cm}
\rule{\textwidth}{0.4pt}


\section{Introduction}

\section{Requirements}

\section{Risk register}

A risk assessment has been performed as part of the initial report to ensure the
team is aware of any problems which could later arise, and to provide a guide as
to how to react when such problems occur. The hazards (risks) have been
identified and classified based upon team members' past experiences in similar
projects and group discussion. The areas these hazards impact were then
analysed, as well as the probability of occurrence. These are then weighted so
that we can identify the risks which are likely to have the greatest detrimental
effect on the project. The likelihood score (LS), impact score (IS) and risk matrix score (RS) are
listed in the table below.

\begin{longtable}[H]{| p{0.6cm} | p{2cm} | p{0.3cm} | p{2.6cm} | p{8.1cm} | p{0.7cm} |}
    \hline
    \cellcolor{titleColor}\textbf{Risk ID}   & \cellcolor{titleColor}\textbf{Risk}                                             &\cellcolor{titleColor}\textbf{LS}        & \cellcolor{titleColor}\textbf{Classification and Possible Impact}                                 & \cellcolor{titleColor}\textbf{Mitigation and Contingency} & \cellcolor{titleColor}\textbf{IS} \\ \hline                                                                                                                                                                                                                                                                                                                                                                                                                                                                                                                                
    \textbf{R.1}   & Short term loss of team members                  & 6       & \textit{Moderate}
\newline Deadline failure                                        
      &  The team can then reactively reallocate the team member's work across remaining team members. To aid with this, the team must proactively ensure that no work relating to the project is outside of team version control. Use of the scrum methodology proactively aids work reallocation, ensuring team members are aware of all assigned work. 
      & 18    \\ \hline
    \textbf{R.2}    & Long term loss of team members                   & 2 & \textit{Catastrophic}
\newline Deadline failure and low standard of deliverables 
    & If a team member is unavailable for an extended period, the team will react by notifying the customer and possible extending deadlines. The proactive procedures mentioned in \textbf{R.1} will also be followed to reduce the impact of this scenario.                                                                                                                                                                                                                                                                                            
    & 10    \\ \hline
    \textbf{R.3}     & Short or long term loss of resources             & 2 & \textit{Catastrophic}
\newline Deadline failure and loss of code base              
    & Proactive use of a source code repository, meaning code-base and history is decentralised. If the repository is lost, the data can be retrieved from the local repository copies and university backups.                                                                                                                                                                                                                                                                                                        
    & 10    \\ \hline
    \textbf{R.4}     & Team member under-performance                    & 3         & \textit{Major}
\newline Deadline failure and low standard  of deliverables            
    & Project plan must be feasible. The skills of the team as a whole, and individual team members must be proactively established early on and taken into account when assigning roles.                                                                                                                                                                                                                                                                                                                                                              
    & 12    \\ \hline
    \textbf{R.5}    & Mis-interpretation of requirements                & 3        & \textit{Major}
\newline Deliverables that are not valid                                        & 
    Requirements, design and implementation strategy must be proactively verified with the customer, this process is iterative, stopping when both customer and developers are content. Any changes to those requirements must result in re-negotiated deadlines.                                                                                                                                                                                                                                                                                    
    & 12    \\ \hline
    \textbf{R.6}    & Slow response to customer queries                & 3        & \textit{Major}
\newline Deadline failure and  low standard of deliverables         
    & Customer has assured a two working day response where possible. Further mitigation can be achieved by proactively communicating issues well in advance of deadlines.                                                                                                                                                                                                                                                                                                                                                                            
    & 12    \\ \hline
    \textbf{R.7}     & Failure to produce required system functionality & 2 & \textit{Catastrophic}
\newline Wasted time and loss of marks                        
    & Customer verification of requirements can counteract this risk. System testing, to ensure all agreed upon requirements are met, will also reduce this risk.                                                                                                                                                                                                                                                                                                                                                                                      
    & 10    \\ \hline
    \textbf{R.8}     & Missing internal team deadlines                  & 5            & \textit{Moderate}
\newline Project falls behind due to missing dependencies                     
    & Perform critical path analysis to identify tasks which will take the longest time and which are a prerequisite to others. A greater team effort can then be assigned to these areas if it seems likely to miss a deadline or halt progress elsewhere.                                           
    & 15    \\ \hline
\end{longtable}


The likelihood score defines probability of something occurring. Utilising
\textit{Kents Words of Estimative Probability}\cite{kent1966strategic}, with
`certain' weighted $7$ and `impossible' weighted $1$.

\begin{longtable}[H]{c c l | c | c | c | c | c | }
		\cline{4-8}
		& & & \multicolumn{5}{ c| }{Impact Score (Least$\rightarrow$Most)} \\ \cline{4-8}
		& & & 1 & 2 & 3 & 4 & 5 \\ \cline{4-8}
		& & & Negligible& Minor & Moderate& Major & Catastrophic \\ \cline{1-8}
		\multicolumn{1}{ |c }{\multirow{7}{*}{Likelihood Score}} & \multicolumn{1}{ |c| }{7} & Certain & 7 & 14 & 21 & 28 & 35 \\ \cline{2-8}
		\multicolumn{1}{ |c }{} & \multicolumn{1}{ |c| }{6} & Almost certain & 6 & 12 & 18 & 24 & 30 \\ \cline{2-8}
		\multicolumn{1}{ |c }{} & \multicolumn{1}{ |c| }{5} & Probable& 5 & 10 & 15 & 20 & 25 \\ \cline{2-8}
		\multicolumn{1}{ |c }{} & \multicolumn{1}{ |c| }{4} & Chances about even & 4 & 8 & 12 & 16 & 20 \\ \cline{2-8}
		\multicolumn{1}{ |c }{} & \multicolumn{1}{ |c| }{3} & Probably not & 3 & 6 & 9 & 12 & 15 \\ \cline{2-8}
		\multicolumn{1}{ |c }{} & \multicolumn{1}{ |c| }{2} & Almost certainly not & 2 & 4 & 6 & 8 & 10 \\ \cline{2-8}
		\multicolumn{1}{ |c }{} & \multicolumn{1}{ |c| }{1} & Impossible & 1 & 2 & 3 & 4 & 5 \\ \hline
\end{longtable}

\begin{table}[h!]
	\begin{tabular}{ | p{2cm} | p{4cm} | p{10cm} | }
		\hline
		\textbf{Score}&	\textbf{Risk Level}&	\textbf{Recommended Response}	\\ \hline
		\textbf{23-35}&	HIGH&	Mitigation plan is required. Immediate action is required.	\\ \hline
		\textbf{11-22}&	MEDIUM&	To be included in the action plan and reviewed.	\\ \hline
		\textbf{0-10}&	LOW&	Included in action plan in limited scope. Minimum review.	\\ \hline
	\end{tabular}
\end{table}


\section{Customer communication}

\section{Glossary}

\begin{description}

	\item[HUMS] Health and usage monitoring system(s).
	\item[Customer] Thales. 
	\item[Consumer] The recipient organisation of the system.
	\item[(End) User] An individual or organisation using the system.
	\item[Client] A piece of computer hardware or software which accesses a
	              service made available by the HUMS system.
	\item[The System] The HUMS we are designing and developing.
	\item[Event] A point of interest flagged up by an analysis system, which
	             may result in a notification.
	\item[Data input client] Anything that provides data to the HUMS through
	                         the input interface.
	\item[Data output client] Anything that receives data from the HUMS
	                          through the reporting or notification interfaces.
	\item[Notification] A message sent by the system as a result of an event
	                    being fired.
	\item[Report] A message sent by the system as a result of a request from a
	              user. 
	\item[HUMS Instance] One installation or occurrence of the system for a
	                     specific consumer.
\end{description}

\vfill
\bibliography{report-refs}
\bibliographystyle{IEEEtran}
\end{document}
