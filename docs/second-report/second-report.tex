\documentclass[10pt,a4paper]{article}

\usepackage[margin=3cm]{geometry}
\usepackage[UKenglish]{babel}
\usepackage{enumitem}
\usepackage{calc}
\usepackage{fancyhdr}
\usepackage{graphicx}
\usepackage{multirow}
\usepackage[table]{xcolor}
\usepackage{float}
\usepackage{longtable}
\usepackage{parskip}
\usepackage{soul}
\usepackage[small,compact]{titlesec}
\usepackage[justification=centering]{caption}

\definecolor{reqColor}{RGB}{80,80,120}
\definecolor{titleColor}{RGB}{138,201,242}
\pagestyle{fancy}
\lhead{T Davies, A Fahie, A Fairbairn, A Free, J Mansfield, R Tucker, M 
Walker}
\chead{}
\rhead{GPIG-C}
\cfoot{\vspace{-0.6cm} \thepage}

\setlist{nolistsep} % Reduces lots of white space around lists

\renewcommand{\headrulewidth}{0.4pt} % Add rules below header
\renewcommand*{\thefootnote}{\fnsymbol{footnote}}

\newcommand{\fr}[1]{\textcolor{reqColor}{\textbf{FR.#1}}}
\newcommand{\frit}[1]{\textit{FR.#1}}
\newcommand{\nfr}[1]{\textcolor{reqColor}{\textbf{NFR.#1}}}
\newcommand{\nfrit}[1]{\textit{NFR.#1}}

\begin{document}

\begin{center}
{\Large GPIG-C Interim Report}

%Word count: \input{wc}\unskip
\footnote{\textit{Using TeXCount, excluding\ldots}}

Friday, 14th February 2014
\end{center}

\vspace{0.3cm}
\rule{\textwidth}{0.4pt}

\section{Glossary}
\label{sec:glossary}

\begin{description}[leftmargin=!,labelwidth=\widthof{\bfseries Data output clientxx}]
	\item[HUMS] Health and usage monitoring system(s).
	\item[The system] The HUMS we are designing and developing.
	\item[Customer] Thales.
	\item[Consumer] The recipient organisation of the system.
	\item[(End) User] An individual or organisation using the system.
	\item[HUMS instance] One installation or occurrence of the system for a specific consumer.
	\item[Client] A piece of computer hardware or software which accesses a service made available by the HUMS.
	\item[Data emitter] Anything that provides data to the HUMS through the input interface.
	\item[Data output client] Anything that receives data from the HUMS through the reporting or notification interfaces.
	\item[Analysis] The processing performed on data within the HUMS.
	\item[Event] A point of interest flagged up by an analysis system, which may result in a notification.
	\item[Notification] A message sent by the system as a result of an event being fired.
	\item[Alerting] The process of generating and delivering notifications.
	\item[Report] A message sent by the system as a result of a request from a user.
	\item[Auditing] The process of generating and returning reports.
	\item[Sensor] A single data-producing source that is monitored by the HUMS through a data emitter.
	\item[System ID] The identifier given to a group of sensors which are, for example, co-located or of the same type.
\end{description}

\section{Introduction}
\label{sec:introduction}

This interim project report details refinements to previously identified requirements of the HUMS and identifies associated important quality attributes. The report also provides a detailed insight into the underlying system architecture from multiple viewpoints, and the implementation and evaluation of a prototype solution based on this design.

\section{Requirements Refinement}
\label{sec:requirements}
After feedback from the initial report the original requirements have been altered and, where possible, refined.

\subsection{Functional Requirements}
\label{sec:requirements-functional}
Below the existing functional requirements are refined, based on customer feedback and the design designs made in the previous report.

\frit{1} is refined to specify how data emitters will send data to the system, and the type of data required. 
\frit{2} required data to be timestamped, but didn't specify how these timestamps were to be produced, \frit{2.1} and \frit{2.2} assign this
responsibility to the data emitter and consumer system, and mandate when timestamps should be created.

\frit{3} is refined by specifying how the data is to be stored, allowing the end user to choose the technology of their choice, meaning the database can be changed without needing to alter other HUMs modules, making it easy to move the HUMS across domains.
\frit{4} was modified such that the previous requirements \frit{4}, \frit{6}, \frit{9}, and \frit{10} can all be expressed as refinements, with the configuration file containing all end user settings. The new \frit{5} and \frit{6} enforce the  functionality that the HUMS must provide as a result of the data in the configuration file.

\frit{7}, previously \frit{12}, is refined to define how data will be analysed for events. It is identified that there must be an API allowing analysis engines, either created by the consumer or included with the HUMS, to extract stored data and produce events. 

\frit{9} to \frit{13} do not appear in the previous report, and were discovered to be necessary following feedback from the customer regarding the ability of the system to allow the output of analysis to effect the monitored system. It was felt that our architecture did not make it clear whether this was possible or not, therefore these requirements were added as clarification.
\begin{description}[leftmargin=1.3cm, itemsep=0.3cm]
	%%FR1
  	\item[\fr{1}]  Data emitters shall be able to push correctly structured data 	to the HUMS.
	\begin{description}
		 \item[\fr{1.1}] The HUMS shall provide an API for data input.

		  \item[\fr{1.2}] The data input API shall require an identifier for
 		 the HUMS being monitored.

 		 \item[\fr{1.3}] The data input API shall require data to be
 		 timestamped.

 		 \item[\fr{1.4}] The data input API shall allow data emitters
 		 to send the IDs of sensors and their values to the HUMS, which will 			then be made available to the analysis engine.

 		 \item[\fr{1.5}] A data emitter, designed to extract data from the given 		test application shall be provided.
	\end{description}
	%%FR2
	\item[\fr{2}]  The HUMS shall allocate a timestamp to new data.
	\begin{description}
	 	 \item[\fr{2.1}]Data input clients shall timestamp data before
 		sending it to the HUMS.
		
 		 \item[\fr{2.2}] Timestamps shall be collected when data is sent 			to storage.
	\end{description}
	%%FR3
	 \item[\fr{3}] The HUMS shall store correctly structured data.
	 \begin{description}
	 	\item[\fr{3.1}] The end user shall integrate their chosen database 			technology with the HUMS
	 	\item[\fr{3.2}] The HUMS shall use a database abstraction layer, 			allowing the database component to be easily changed according to 			need.
	  \end{description}
	%%FR4
	 \item[\fr{4}] The HUMS shall an store end user system configuration 			file.
	 \begin{description}
	  	\item[\fr{4.1}] The HUMS shall allow authorised users to modify
 		configuration files. 
		 \item[\fr{4.2}] The HUMS shall allow the consumer to define a 				low storage threshold.
		  \item[\fr{4.3}] The HUMS shall allow the consumer to set an expiry 			time on data.
		  \item[\fr{4.4}]  The HUMS shall allow the user to define that, 				upon reaching their defined data storage quota, new data is no 				longer added.
		 \item[\fr{4.5}] The HUMS shall allow the user to de- fine that, upon 			reaching their defined data storage limit, old data is deleted to make 			room for new data.
	\end{description}
	%%FR5
	 \item[\fr{5}] The HUMS shall send a notification when the consumers 		defined low storage threshold is reached.
	  \item[\fr{6}] The HUMS must store no more data records than the 			consumers defined storage quota.
	\item[\fr{7}]Events shall be triggered in response to new data that 			matches a pattern that the consumer has specified.
	  \begin{description}
	  
		 \item[\fr{7.1}]  The HUMS shall allow the consumer to specify what 		 	patterns of data will produce events.
		  
		 \item[\fr{7.2}] The HUMS shall allow the user to define their
 		own events.
		
 		\item[\fr{7.3}] The HUMS shall provide an analysis API, allowing 			analysis engines to extract and analyse data.
		
		\item[\fr{7.4}] Events shall be created by analysis engines.

 		\item[\fr{7.5}] The HUMS shall provide a simple analysis engine
 		in the form of a rules engine.
	\end{description}
	
	\item[\fr{8}]After the sending of a notification for an event of a particular 		type, no more notifications for an event of that type will be sent 				during the cool down period.
	\item[\fr{9}]The system shall dispatch an event to a notification generator 		when a specified data pattern in detected.
	\item[\fr{7}] The HUMS must interface with reports engines, allowing 			them to pull reports.
	\item[\fr{11}] The system shall provide basic examples of notification 			engines to work with the test application.
	\item[\fr{12}] The system shall allow for notifications to change the way in 		which data is sensed.
	 \item[\fr{13}] The system shall allow notifications to be sent back to the 		system being monitored.
\end{description}

\subsection{Non-Functional Requirements}
Non-function requirements remain remain the same, however, \nfrit{14} has been added in order to clarify that data can be sent and stored in any format, enforcing that the consumers system will not have to conform to the HUMS. It is also recognised that \nfrit{5}, \nfrit{9} and \nfrit{11} are requirements specific to the prototype, as functional requirements require the datastore to be changeable meaning, in the general case, requirements about the abilities of the datastore are not verifiable.
\begin{description}
\item[\nfr{1}] The HUMS shall be able to receive hardware changes without loss of previously stored data.
\item[\nfr{2}]  Users shall be provided with documentation detailing how to use the system.
\item[\nfr{3}] When running externally on servers, the system must be accessible to end users in multiple
geographic locations.
\item[\nfr{4}]  The system shall only accept data from an input client providing valid credentials. 
\item[\nfr{5}] The system shall store data according to the relevant industry security standards. 
\item[\nfr{6}]  The system shall be tested to ensure all requirements are met before deployment.
\item[\nfr{7}] The customer will complete acceptance testing before the system is deployed.
\item[\nfr{8}] The system must be able to support at least 5 output clients per HUMS instance. 
\item[\nfr{9}]  The system must cope with up to 2000 data input requests per second per HUMS instance. 
\item[\nfr{10}] The system must be available for no less than 99.9\% of each month.
\item[\nfr{11}]  Data must be backed up within 24 hours of having been made available to the system.
 \item[\nfr{12}] Timestamps applied by the system must be accurate to within 5ms of UTC.
\item[\nfr{13}]  The system shall dispatch notifications within 5ms of the event being triggered.
\item[\nfr{14}] The system shall support storing data without requiring specific schemata.
\end{description}


\section{System Architecture}
\label{sec:architecture}

\subsection{Quality Attributes}
\label{sec:architecture-quality}

Quality attributes are defined in order to ensure the architectural decisions 
made when creating the system architecture not only provide the required 
functionality, but also produce a system meeting the expectations of all 
stakeholders. For example the system may be expected to meet certain 
security standards or availability levels. Explicitly defining these quality 
attributes before concretely designing a system allows for the correct 
tactics and patterns to be adopted in order to maximise the utility of important 
qualities. When examining the HUMS non-functional requirements the 
following system quality attributes were identified:
	\begin{center}
	\textit{Flexibility, Performance, Modifiability, Security, 
Testability, Usability}
	\end{center}
For the prototype implementation, at this stage, usability and security 
have not been heavily considered, however in the real system they 
would be of high importance. Security is a key aspect of any nontrivial HUMS, 
as data being monitored must be both secret and safe. It must be impossible 
for an unauthorised user to either insert or remove data from the system, and 
so all data would need to be stored according to relevant security standards 
(e.g., encryption and password protection for personal data covered by the 
Data Protection Act). Additionally, data in transit between emitters and the data 
input API must be protected, which can be achieved using a symmetric-key 
encryption system, which is well supported by many programming languages. 
Determining who has the rights to access which data would fall under the 
purview of an individual customer, who would manage access rights using the 
configuration front-end of our system.

Flexibility is important to the HUMS as it is too be used across multiple 
domains, meaning it cannot be to specific. Following a plugin 
architecture means the system can remain general and flexible however 
can include domain specific elements, such as analysis engines, 
creating a software product line where the core of the HUMS can be 
shared across all domains.

Modifiability is also an important qualities of the HUMS, though not 
measurable, lack of modifiability can increase the cost and time taken to 
complete the project. The HUMS will initially be built to tackle a single 
domain (software), however will in the future be required to work on embedded 
systems, mechanical systems, electrical systems and even people. In 
order to ensure these changes are as efficient as possible the initial system 
must be modifiable. Tactics that can be used to increase the modifiability 
of the system include using small modules within the system, decreasing the 
coupling between those modules and increasing the cohesion. In order 
to achieve low coupling  the HUMS uses interfaces between modules and 
restricts dependencies, such that the majority of modules can be 
swapped out without large changes to the overall system.

The performance of a system is normally measured by its latency and 
throughput. The HUMS is required to dispatch notifications through the 
notification generator after events have been triggered by the analysis 
controller, with a latency less than 5ms, the system is 
also required to be able to support multiple data output clients. Tactics 
such as reducing overhead though careful use of data structures and prioritising events can help reduce latency, at least for the most important events, defined by the end user. It may also be possible to add concurrency, such that multiple notification threads are used when demand for notifications and reports are high, reducing the bottleneck created at the notification generator.

In in order to ensure the HUMS is testable we followed the test driven 
development methodology where possible, writing and planning the tests 
before writing the code, passing tests means code performs the 
expected functionality. In addition to this we also ensured defined how 
each requirement of the system would be tested and created a 
comprehensive test plan, detailing integration, system, unit and 
acceptance testing.

\section{System Views}
\label{sec:architecture-views}

When designing the HUMS we identified four different views of of the
system which must be documented, taking inspiration from the Hofmeister approach to software architecture:  %%CITE Hofmeister, Nord & Soni, Applied Software Architecture, Addison Wesley.

\begin{description}
  \item[Module View] The required software modules and relationships
    between them at a high level.

  \item[Behavioural View] The flow of data and dynamic actions within
    the system.

  \item[Execution View] How modules are geographically positioned.

  \item[Conceptual View] Class diagrams, showing a concrete
    implementation of the functionality.
\end{description}

\subsection{Module View}
\label{sec:architecture-moduleview}

Having determined the quality attributes of the HUMS and the tactics
which can be employed in order to achieve a high utility of these
qualities, a conceptual view can be formed. The conceptual diagram
shown in figure \ref{fig:ComponentDiagram} depicts the modules of the
HUMS and the interactions between these modules at a high level.

%%TODO this diagram doesn't show the core or have a key

\begin{figure}[!ht]
  \centering
  \includegraphics[width=14cm]{images/ComponentDiagram.png}
  \caption{The module diagram of the HUMS}
  \label{fig:ComponentDiagram}
\end{figure}

\begin{description}
  \item[Core] A collection of modules that must be run on one system, including the Data Input Layer, Analysis Controller, Data Abstraction Layer and Notification Generator. Additional modules can be integrated on the same system or connected over a network interface.

  \item[Data Emitter] Sends data to the system core in a standard
    format. May be included within the HUMS or created by the end
    user.

  \item[Data Input Layer] Receives data from a data emitter and sends
    it to the data abstraction layer. Alerts the analysis controller
    that new data was received.

  \item[Data Abstraction Layer] Handles all interaction with the
    database, such that if the database was to be altered, only this
    layer would need to change.

  \item[Analysis Controller] Alert the analysis engines when new data
    has arrived, pulling the required data from the data abstraction
    layer. If an analysis engine determines an event has occurred
    within the data then the analysis controller forwards that event
    to the notification generator.

  \item[Analysis Engine] Analyses the data looking for trends, if a
    trend is found an event is returned to the analysis
    controller. Analysis engines may be included in the HUMS or
    created by the end user.

  \item[Notification Generator] Receives events from the analysis
    controller, defines an abstract notification and sends it to the
    appropriate notification engine.

  \item[Notification Engine] Receives an abstract notification from
    the notification generator and creates a concrete notification,
    for example an email or triggering a change in system
    behaviour. Notification engines may be included in the HUMS or
    created by the end user.

  \item[Reports Engine] Pulls data from the data abstraction layer and
    formats it into a report, for example a PDF or graph. Reports
    engines may be included in the HUMS or created by the end user.
\end{description}

\subsection{Behavioural View}

The module view shows how modules of the system are connected, however 
does not fully represent the interactions between components and the dynamic actions of the system. A behavioural view can be used to detail this information, showing the flow of data and events within the HUMS. Figure \ref{fig:CommunicationDiagram} shows a communication diagram of the 
HUMS, detailing how the HUMS goes from receiving data from a data emitter to generating a notification. The different objects in the system are shown, along with all the interactions and links between them. The messages sent between different parts are shown and labelled with numbers to represent the chronological order in which they occur.

\begin{figure}[!ht]
  \centering
  \includegraphics[width=14cm]{images/CommunicationDiagram.png}
  \caption{Communication diagram showing a behavioural view of message 
flow
through the system, using UML 2 notation}
  \label{fig:CommunicationDiagram}
\end{figure}

%%Input Diagram WITH KEY

\subsection{Execution View}
With view to deployment and execution, the HUMS needs to be tailorable, the core system may be running on the same hardware as the data emitter, it may also be geographically distant, engines may also be on separate machines or may all be together. Figures \ref{fig:DeploymentDistributed} and \ref{fig:DeploymentIsolated} show the main scenarios for deployment of the HUMS, with a view that if the system architecture can perform under these situations, then it can perform under all others.

Figure \ref{fig:DeploymentDistributed} shows abstractly how the consumer system could be geographically distant to HUMS core, engines and datastore. It also shows how a remote admin centre could be used, allowing configuration files to be created at a high level by business personal, as opposed to developers. This deployment possibility gave the team the idea of creating a system as service (SaaS), where multiple customers can simultaneously use a single storage, analysis, and notification server. 
This is shown in figure \ref{fig:DeploymentService}, and is made possible by associating sensor data with a system ID.
For administrating the HUMS as a service, a hosted admin
centre will be created, which allows users to specify analysis and reporting behaviours using a simple rules engine-like interface. Users can define their own events and how they should be handled, view all of the registered input clients, pull reports, and view live graphs of data.

Figure \ref{fig:DeploymentIsolated} shows we we envision the prototype HUMS being deployed, with all modules on the same machine. It also shows how the system will be deployed on an embedded system, with configuration files being created manually, without an admin centre, and  all module existing on the same device.
\begin{figure}[!ht]
  \centering
  \includegraphics[width=10cm]{images/DeploymentDistributed.png}
  \caption{Deployment diagram showing HUMS spread across multiple 
systems}
  \label{fig:DeploymentDistributed}
\end{figure}

\begin{figure}[!ht]
  \centering
  \includegraphics[width=12.5cm]{images/DeploymentIsolated.png}
  \caption{Deployment diagram showing HUMS on one system}
  \label{fig:DeploymentIsolated}
\end{figure}

\begin{figure}[!ht]
  \centering
  \includegraphics[width=10cm]{images/DeploymentService.png}
  \caption{Deployment diagram showing HUMS as a multi-user service}
  \label{fig:DeploymentService}
\end{figure}

\subsection{Conceptual View}
The Conceptual View is a description of the the entire system in terms of it's individual components, for example, classes in an object-oriented design.

\subsubsection{Data Input Layer}

The data input layer, shown in figure \ref{fig:dataInputLayer},
contains the following classes:

\begin{description}
  \item[DataInputServer] An object which starts up a multithreaded
    server to receive connections, and which periodically processes
    received data by sending it off to the database and notifying the
    analysis component. It accesses these by having references to them
    passed in upon creation.

  \item[ProtoReceiver] Listens on a socket for inbound connections,
    and then hands them off to a new ProtoMultiReceiver, running in a
    new thread, for processing. It also maintains a thread-safe
    concurrent queue, which all of the receivers write to, and the
    server reads (and removes) from. This use of multithreading
    enables multiple clients to send data at the same time easily, and
    the use of a queue ensures that data is inserted into the database
    in the order in which it is received.

  \item[ProtoMultiReceiver] Receives data from a particular data input
    client, writing received data to the queue.
\end{description}

\begin{figure}[ht!]
  \centering
  \includegraphics[width=6.2cm]{images/DataInputLayer/DataInputServer.png}
  \includegraphics[width=8.3cm]{images/DataInputLayer/ProtoReceiver.png}
  \includegraphics[width=9.5cm]{images/DataInputLayer/ProtoMultiReceiver.png}
  \caption{Conceptual diagrams of the data input layer classes, using 
UML 2 notation}
  \label{fig:dataInputLayer}
\end{figure}

\subsubsection{Data Abstraction Layer}

The data abstraction layer, shown in figure
\ref{fig:dataAbstractionPackage}, contains the following classes:

\begin{description}
  \item [System Data Gateway] An interface used to abstract the
    datastore implementation details from the rest of the
    application. An implementation of this interface will provide ways
    to read and write from the chosen datastore, meaning the rest of the
    system is not concerned with the datastore implementation. When
    interfacing with a datastore the end user will be required to
    implement this interface.

  \item [EmitterSystemState] An object representing the data received
    from a data emitter at a particular time. The state object holds
    the ID of the end user's system and the creation timestamp along
    with a map of sensor IDs against the value of the sensor at that
    particular time. A map was used to represent the sensors and
    values so that the end user is free the send data from as many or
    as few sensors at a time, reducing the number of writes to the
    database. For serialisation this object includes methods to
    convert to and from JSON.

  \item [DataJSONAttribute] An enumeration that wraps up the keys used
    to look up values in the JSON serialisations of the
    EmitterSystemState and QueryResult classes.

  \item [SensorState] An object representing the state of a particular
    system sensor. A sensor has an ID, a value and two timestamps, one
    specifying the time the sensor value was added to the database and
    the other the time the value was created.

  \item [QueryResult] An object representing the result of a database
    query, containing the ID of the end user's system to which the
    data relates and a list of sensor states pulled from the database.
\end{description}

\begin{figure}[ht!]
  \centering
  \includegraphics[width= 9.5cm]{images/DataAbstractionLayer/systemDataGateway.png}
  \includegraphics[width= 5cm]{images/DataAbstractionLayer/dataJsonAttribute.pdf}
  \includegraphics[width= 7cm]{images/DataAbstractionLayer/queryResult.pdf}
  \includegraphics[width= 7cm]{images/DataAbstractionLayer/sensorState.pdf}
  \includegraphics[width= 9cm]{images/DataAbstractionLayer/emitterSystemState.pdf}
  \caption{Conceptual diagrams of the data abstraction layer classes, 
using UML 2 notation}
  \label{fig:dataAbstractionPackage}
\end{figure}

\subsubsection{Analysis Component}

The analysis component, shown in figure
\ref{fig:dataAnalysisComponent}, contains the following classes:

\begin{description}
  \item [Analysis Controller] An object that coordinates all analysis of
    sensor data. The object interfaces with the data input layer,
    receiving a system ID when analysis is required. Once analysis has
    been carried out the controller processes the results and triggers
    notifications where appropriate. Analysis engines are associated
    with the analysis controller by class loading allowing for users to
    add and remove analysis engines at run-time.

  \item [Analysis Engine] An abstract class that implements the basic
    functionality and defines the methods required of an implemented
    analysis engine. Each engine has a list of associated systems. The
    implementation of the analyse method interfaces with the database
    abstraction layer to retrieve the appropriate sensor data and then
    performs the desired analysis. Once analysis is complete the engine 
    returns a result object. When creating their own analysis 
    engines the end user will be required to extend this class, implement 
    the abstract methods and define what constitutes an event.

  \item [Mean Analysis] An object that extends the analysis engine
    class and implements mean analysis. When called by the analysis
    controller, sensor data is retrieved for all associated systems
    and mean analysis is performed. If a mean value fall outside of
    the acceptable bounds then a notification is raised.

  \item [Result] An object representing the result of the analysis
    carried out by the chosen analysis engine. The object contains a
    map of any data to be saved back to the database and a flag
    indicating whether a notification needs to be triggered. For
    serialisation the data to be saved to the database is stored in a
    map of string key-values. When implementing their own analysis
    engine the end user will be able to specify what data is to be
    saved back to the database.

  \item [Event] An object representing an event,
    containing the name of the analysis engine that triggered the
    event, the ID of the system that the event applies
    to and the result object from the analysis engine that triggered
    the event.
\end{description}

\begin{figure}[ht!]
  \centering
  \includegraphics[width= 9.5cm]{images/Analysis/AnalysisController.png}
  \includegraphics[width= 5cm]{images/Analysis/Result.png}
  \includegraphics[width= 12cm]{images/Analysis/AnalysisEngine.png}
  \includegraphics[width= 6cm]{images/Analysis/Event.png}
  \includegraphics[width= 8cm]{images/Analysis/MeanAnalysis.png}
  \caption{Conceptual diagrams of the analysis component classes, using 
UML 2 notation}
  \label{fig:dataAnalysisComponent}
\end{figure}

\subsubsection{Notification Generator}

\begin{figure}[ht!]
  \centering
  \includegraphics[width= 7cm]{images/Notification/NotificationGenerator.png}
  \includegraphics[width= 5cm]{images/Notification/NotificationEngine.png}
  \includegraphics[width= 4.5cm]{images/Notification/EmailNotification.png}
  \caption{Conceptual diagrams of the notification component classes, using 
UML 2 notation}
  \label{fig:notificationComponent}
\end{figure}

\section{Interim Prototype Implementation}

\subsection{Technologies}

We chose to develop the core of the HUMS prototype in Java. Java can
be compiled to bytecode that can be executed directly in all
environments that support a Java Virtual Machine, including various
operating systems and the x86, x86-64 and ARM CPU architectures,
making it highly portable between both desktop, server and embedded
systems. % REF: http://www.oracle.com/technetwork/java/javase/config-417990.html

Whilst there may exist deployment platforms that are not supported, at
this stage, we do not see this as a problem since the underlying
architecture of the system is transferable across languages.

All members of our team were familiar with the Java programming
language and thus no time was spent having to get the team up to speed
on a new language. We did not impose a language restriction when
developing the data emitter and various engines as the data being
passed between them and the core are language independent, meaning 
end users are free, even with the prototype, to implement engines in any
language they see fit. For the prototype we used both Java and
JavaScript when creating the engines, languages team members are
familiar with, allowing us to show off the versatility of the system.

Upon determining to monitor CPU and heap usage, we began to investigate
technologies that would allow us to do so, rather than reinvent the wheel.
Doing this also had the benefit of providing us with established, well-tested,
tools with which we could easily implement data input clients for further
systems. As the test program is a Java program, we had to connect to its JVM to query the heap usage, as the operating system would just report the size of the JVM heap. We settled on using SIGAR for process monitoring, and JMX for monitoring heap usage.

To communicate between the code and the other modules, we serialised objects
to send them over a network connection. We used JSON and Protocol Buffers.
For connections that require data to be serialised to text, we used JSON since
it is human readable and maps well to our internal representation of objects
in Java. To read and write JSON we used the Jackson library, since it offers
a high-performance streaming API. When there was no such restriction on the 
format of the transmitted data, we used Google Protocol Buffers since they offer 
high-performance as is necessary for the potentially high-volume connections 
between the Core and the other modules, such as the Data Emitter and the Analysis Engines.

We chose to use Google App Engine to host the prototype's database
module, displaying how easily it can incorporate existing, popular
technologies as plugins. The prototype could easily be extended to
work with other database technologies simply by implementing an
interface within the data abstraction layer.

We created a prototype admin centre to show how the HUMS could be
managed remotely, viewing data and modifying settings online, through
a web browser, whilst running as a service. The admin centre was
developed using the Bootstrap front-end web framework, including 
HTML, JavaScript and CSS, in order to create a prototype realistically
simulating an interface to the HUMS that is accessible on desktop
computers, laptops and mobile devices alike.


We used various Java libraries to support our prototype, saving us 
development time and providing a robust codebase. For testing we used 
the Mockito framework to allow us to verify elements of our codebase in 
isolation, by offering a concise way to mock out dependancies. To 
interrogate the state of the test application provided by the Customer, we 
used the Sigar cross-platform system information API and JMX to query the 
vital parameters of the process, including CPU utilisation and memory 
footprint.

We used Git for version control, allowing the entire team to
contribute to the project simultaneously, whilst keeping a reversible
history of all alterations made by each person.

\subsection{Current Functionality}

The provided data emitter looks for a JVM executing the sample application, and
begins recording its CPU and memory usage, reporting to the data input API
every second. A period of four seconds are allowed to elapse before the first
report, in order for the CPU usage readings to stabilise.

The data input system as currently implemented supports multiple simultaneous
connections from input clients, although we did not investigate how many the
prototype could handle. Upon receipt, the data is put into a queue, which is
periodically processed by the data input server, by adding each entry to the
database and notifying the analysis engine.

\subsection{Testing} 
We created and implemented a test plan that laid out our key ideas
with respect to verifying our that solution fulfilled its
requirements. In this section we given the mapping between each
test and its associated requirement. We
defined our plan, where appropriate, in terms of each module within
the system and what specific, testable attributes that module must
have to fulfil its requirements. We also further deconstructed the
testing into various levels of abstraction, mapping the higher level
requirements to their lower-level requisite requirements: acceptance
testing, system testing, integration testing and unit
testing.

% Does it make sense to have system/acceptance testing within a
% module? Given we are using a plugin architecture, I am not sure. TD

% TODO We probably need something about overall system & acceptance
% testing. These are mainly the white/black/grey box tests in the
% first report, e.g. NRF.9, "The system must cope with up to 2000 data
% input re- quests per second per HUMS instance."

\begin{description}

  \item[Data Emitter] As the purpose of the data emitter is to run
    continuously, monitoring a system and reporting on it, thereby
    fulfilling \frit{1.2}. We completed inspection testing to verify 
    that data was reported, along with running the program for 
    several hours to verify that it continued to function.

  \item[Data Input Layer] For the data input layer, it is essential that large
    amounts of data can be received in parallel. We verified this by
    inspection testing, and also by running multiple clients simultaneously and
    verifying that the data from both was recorded, thus verifying \frit{1.2}
    and \nfrit{9}.

  \item[Data Abstraction Layer] Given the high throughput and
    importance of storage to the function of the HUMS as a whole, the
    data abstraction layer's key attributes are availability and
    performance. We completed an inspection test and used the 
    GWTSystemDataGatewayTest class to verify \frit{3.2}. At the unit 
    testing level, as  part of TDD, we focussed on the comparably complex,
    well defined module-internal functions, such as the generation and 
    parsing of the JSON serialisation of objects.
    
  \item[Analysis Controller] % No tests written for this
  
  \item[Analysis Engine]
  For the analysis engine, we created a unit test to check each aspect of the 
  engine as per \frit{7.1} and \frit{7.4}. We utilised Mockito, which is a framework 
  for testing which allows mocking of results from methods of a dependancy 
  class to restrict the possible source of failure of a test as narrow as possible. 
  This was especially useful for potentially flakey tests, such as connecting over 
  a network that may not be available at test time and database access. Our unit 
  tests covered valid, extreme, missing values were sent to the analysis engine.

  \item[Notification Generator] % No tests written for this
  
  \item[Notification Engine] % No tests written for this
  
  \item[Reports Engine] % No tests written for this
  
 \end{description}
 \subsubsection{Traceability Matrix}

\subsection{Evaluation on Test Application}
\label{sec:prototype-evaluation}
% connects to JVM using JMX, launches monitoring tool and monitors JVM memory, cpu usage (processmonitor.java) checks total cpu time of the process based upon difference in time between two time intervals, 
In order to evalute or system with respect to the provided test application, 
we have created a monitoring package for our prototype specifically for the 
test application. We are utilising the Java Management Extensions (JMX) 
Technology to obtain a measure of the memory utilised by the application. 
We have also created a process monitoring class which monitors the total
 cpu usage of the application. This is achieved by 
 measuring the difference in total cpu usage time between two points in 
 time. These results are then taken, and can be graphed and monitored.


\section{Team Structure}
\label{sec:team}

As in the initial report, we use the following acronyms for team members:

\begin{tabular}{ p{4cm} p{4cm} p{4cm} p{4cm} }
  \textbf{AJF}-Adam Fahie &
  \textbf{AIF}-Andrew Fairbairn &
  \textbf{TF}-Anthony Free &
  \textbf{TD}-Tom Davies \\
  \textbf{JM}-Joseph Mansfield &
  \textbf{RT}-Rosy Tucker &
  \textbf{MW}-Michael Walker & \\
\end{tabular}

Before beginning the second stage, we ascertained which modules everyone had in
the second half of the Autumn and the first half of the Spring terms, which we
then considered as we assigned tasks to people.

\subsection{Strengths and Weaknesses}
\label{sec:team-strengths}

To aid in deciding which technologies to use in the design and implementation
of the solution, we established what the strengths and weaknesses of each team
member were,

\begin{longtable}[H]{|p{2cm}|p{8cm}|p{5cm}|}
  \hline \cellcolor{titleColor}\textbf{Initials} &
  \cellcolor{titleColor}\textbf{Experience} &
  \cellcolor{titleColor}\textbf{Weaknesses}\\

  \hline AJF
  & Java, C, Ruby, Spring, Hibernate, Gradle, Design Patterns, Tomcat, JBoss,
  Git, SQL
  & Web Development, NoSQL, UI/UX, Technical Writing \\

  \hline AIF
  & Java, C++, Python, Javascript, PHP, UI/UX, Web Development, NoSQL, SQL
  & Spring, Technical Writing \\

  \hline TF
  & PHP, Javascript, Java, VB, C, Linux, MDE Developer, Git, SQL
  & UI/UX, Technical Writing \\

  \hline TD
  & Java, C, Python, Scala, Android, Swing, Machine Learning, SQL
  & NoSQL, Web Development, Git \\

  \hline JM
  & C, C++, Java, Python, UI/UX, Android, Web Development, Git, SQL
  & Hardware, NoSQL \\

  \hline RT
  & Java, C, Objective C, Javascript, Android, iOS, Swing, Design Patterns,
  Google AppEngine, HCI, NoSQL, SQL
  & Spring, Git \\

  \hline MW
  & C, Java, Python, Git
  & UX/UX, Technical Writing, Concurrency \\
  \hline
\end{longtable}

As can be clearly seen, everyone in the team is proficient with Java, and thus
it was a natural implementation choice for the prototype.

\subsection{Developmental Subteams}
\label{sec:team-subteams}

For the development of the prototype we split into three subteams,
concentrating on the different areas of the system. We then collaborated to
produce the reporting backend.

We decided to divide the labour in this way, rather than all work on the entire
prototype simultaneously, as it facilitates specialisation and discussion,
allowing people to spend a long time thinking of the best way to implement a
particular feature, without the worry that the rest of the prototype may suffer
as a result of this diversion.

\begin{description}
  \item[Team Sense] implemented the data emitter for the test application,
    by producing reusable monitoring components. They also implemented the data
    input API as a multithreaded server, in order to meet requirements
    regarding simultaneous usage. (JM and MW)

  \item[Team Store] implemented the database abstraction interface, and also a
    concrete implementation using Google Appengine as a backing store. The API
    was kept generic in order to meet requirements regarding flexibility of
    database implementation. (TD and RT)

  \item[Team Analyse] implemented the analysis controller, and the analysis
    engine API. The engine API was implemented to be very sparse so that a large
    number of different concrete implementations could be supplied. (AJF, AIF,
    and TF)
\end{description}



\section{Risk register}
\label{sec:riskregister}

A risk assessment has been performed as part of the interim
report to ensure the team is aware of any problems which could later
arise, and to provide a guide as to how to react when such problems
occur. The hazards (risks) have been identified by examining the 
current literature \cite{boehm1991software,jones1998minimizing}, drawing upon team members' past experiences and group 
discussion. Classification of the identified hazards, their impact on the system and their probability of occurrence is performed 
in accordance with the Risk Management Guide for Information Technology 
Systems \cite{stoneburner2002risk}. The areas these hazards impact were then analysed, as 
well as the probability of occurrence. These are then weighted so 
that we can identify the risks which are likely to 
have the greatest detrimental effect on the project. The likelihood 
score (LS), impact score (IS) and risk matrix score (RS) are listed 
in the table below.

\begin{longtable}[H]{| p{0.7cm} | p{2cm} | p{0.3cm} | p{0.3cm} | p{2.4cm} | p{4.1cm} | p{3cm} | p{0.4cm} |}
  \hline
  \cellcolor{titleColor}\textbf{Risk ID} &
  \cellcolor{titleColor}\textbf{Risk} &
  \cellcolor{titleColor}\textbf{LS} &
  \cellcolor{titleColor}\textbf{IS} &
  \cellcolor{titleColor}\textbf{Impact Description} &
  \cellcolor{titleColor}\textbf{Mitigation} &
  \cellcolor{titleColor}\textbf{Contingency} &
  \cellcolor{titleColor}\textbf{RS}\\
  
  \hline \textbf{R.1}
  & Loss of team member(s)
  & 6
  & 3
  & internal and/or deadline failure
  & Ensure that no work relating to the project is outside of team
  version control. Use of scrum methodology proactively aids work
  reallocation, ensuring team members are aware of all assigned work.
  & Reallocation of work across remaining team members, may have to notify customer and possible deadline extension.
  & 18 \\
  
  \hline \textbf{R.2}
  & Personnel under performance
  & 6
  & 3
  & internal and/or deadline failure
  & Team strengths and weaknesses identified and work assigned appropriately, slack built into the schedule to account for under performance
  & Reallocation of work across remaining team members, may have to notify customer and possible deadline extension.
  & 18 \\
  
  \hline \textbf{R.3}
  & Time pressures limiting amount of development time
  & 
  & 
  & Missing deadline, not fulfilling requirements, low quality deliverables
  & Define scope of work early, commit to complete requirements and allocate work to sprints 
  & Reduce functionality required and/or notify customer of later delivery
  & \\
  
  \hline \textbf{R.4}
  & Developing user interface that does not satisfy users requirements
  & 
  & 
  & Poor user experience, loss of confidence in the solution
  & Continuous, iterative development process, regularly incorporating users feedback on interface
  & Redevelop user interface
  & \\
  
  \hline \textbf{R.5}
  & Adding more functionality than necessary
  & 
  & 
  & Additional, unnecessary functionality, possible deadline miss
  & Ensure a clear, shared vision of the prototype system and collective ownership with open communication
  & 
  & \\
  
  \hline \textbf{R.6}
  & Deprecation of required Google App Engine functionality
  & 
  & 
  & Unable to deliver required functionality
  & Research pending deprecations to ensure required functionality will be supported
  & Move to another key-store database (relatively simple)
  & \\
  
  
  \hline \textbf{R.7}
  & Comms link to Google App Engine fails
  & 
  & 
  & Loss of service to users, loss of confidence in the system
  & A final commercial version would use the premium Google service which guarantees an uptime of 99.9\% (\nfrit10)
  & Additional data abstraction layer could be built to interface with another database service
  & \\
  
  \hline \textbf{R.8}
  & Data interception between our application and App Engine 
  & 
  & 
  & Loss of confidence in the system
  & A final system would encrypt traffic between our system and App Engine, currently just sending (non-sensitive) test data
  & Additional penetration testing to identify security flaws
  & \\
  
  \hline \textbf{R.9}
  & Data write/read times to/from App Engine too slow
  & 
  & 
  & Create a task backlog: gradual increase in load until system
  is non-operational (failure to meet capacity requirement \nfrit9)
  & Load testing of system throughout development to ensure \nfrit9
  & Re-engineering of system, possible need to switch datastore
  & \\
  
  \hline \textbf{R.10}
  & Implementation does not facilitate data storage for a variety of systems
  & 
  & 
  & Failure to meet key modifiability requirements 
  & Adoption of a key-value database allows us the flexibility to store a variety of data
  & Examination of data we are struggling to store and implement additional data gateway for it
  & \\  
  
  \hline \textbf{R.11}
  & System performance required unattainable
  & 2
  & 4
  & Could cause 
  & Choice of implementation or choice of utilised hardware must offer
  enough in the way of performance to facilitate acceptable system
  performance.
  & All of, or combination of: optimisation of code, use of faster
  hardware, use of faster networks
  & 8 \\    
  
  \hline \textbf{R.12}
  & Stock analysis components computing incorrect values
  & 
  & 
  & Failing to meet fundamental requirements, customer loses confidence in the system
  & Adopting a test driven approach to development throughout the development process
  & Re-engineering of failing components
  & \\
  
  \hline \textbf{R.13}
  & Analysis component computation too slow
  & 
  & 
  & Creates an analysis task backlog: gradual increase in load until system
  is non-operational
  & Load testing of analysis component throughout development, code optimisations 
  & Re-engineering of analysis components
  & \\  
  
  \hline \textbf{R.14}
  & Notification computation too slow
  & 
  & 
  & Creates a notification backlog: gradual increase in load until system
  is non-operational
  & Load testing of notification component throughout development, code optimisations 
  & Re-engineering of notification system
  & \\
  
  \hline \textbf{R.15}
  & GitHub becomes unavailable for a prolonged period of time
  & 
  & 
  & Difficulty merging work from different developers machines
  & Distributed nature of Git means little work would be lost
  & Host our own git server within the Computer Science Department
  & \\ 
  
  \hline \textbf{R.16}
  & Malicious destruction of code repository on GitHub
  & 
  & 
  &
  & Two-step authentication used on GitHub, distributed nature of Git means little work would be lost
  & Host our own git server within the Computer Science Department
  & \\    
    
  \hline
\end{longtable}       

The likelihood score defines probability of something occurring. Utilising
\textit{Kents Words of Estimative Probability}\cite{kent1966strategic}, with
`certain' weighted $7$ and `impossible' weighted $1$.

\begin{longtable}[H]{|c|c|l|c|c|c|c|c|}
  \cline{4-8} \multicolumn{3}{c|}{} & \multicolumn{5}{ c| }{Impact Score (Least$\rightarrow$Most)} \\
  \cline{4-8} \multicolumn{3}{c|}{} & 1 & 2 & 3 & 4 & 5 \\
  \cline{4-8} \multicolumn{3}{c|}{} & Negligible& Minor & Moderate& Major & Catastrophic \\

  \hline \multirow{7}{*}{Likelihood Score} & 7 & Certain & 7 & 14 & 21 & 28 & 35 \\

  \cline{2-8} & 6 & Almost certain & 6 & 12 & 18 & 24 & 30 \\
  \cline{2-8} & 5 & Probable& 5 & 10 & 15 & 20 & 25 \\
  \cline{2-8} & 4 & Chances about even & 4 & 8 & 12 & 16 & 20 \\
  \cline{2-8} & 3 & Probably not & 3 & 6 & 9 & 12 & 15 \\
  \cline{2-8} & 2 & Almost certainly not & 2 & 4 & 6 & 8 & 10 \\
  \cline{2-8} & 1 & Impossible & 1 & 2 & 3 & 4 & 5 \\
  \hline
\end{longtable}

\begin{longtable}[H]{ | p{2cm} | p{4cm} | p{8.5cm} | }
  \hline
  \cellcolor{titleColor}\textbf{Score} &
  \cellcolor{titleColor}\textbf{Risk Level} &
  \cellcolor{titleColor}\textbf{Recommended Response} \\

  \hline \textbf{23-35} & HIGH & Mitigation plan is
  required. Immediate action is required.\\

  \hline \textbf{11-22} & MEDIUM & To be included in the action plan
  and reviewed.\\

  \hline \textbf{0-10}& LOW & Included in action plan in limited
  scope. Minimum review. \\
  \hline
\end{longtable}

\section{Customer communication}
\label{sec:communication}

Our initial communication with the customer for this stage of the project was in the form of a face to face meeting where we received feedback on our previous report. This feedback was recorded and acted upon, including extending the risk register to include more technical risks and examining the qualities of the system. Our next contact was through email in which we sent a low fidelity prototype of the HUMS admin centre, explaining our idea of hosting a HUMS as a SaaS or PaaS, to which the customer replied the idea was interesting and suggested some small improvements. 

After each communication with the customer we organised group meetings within two days to discuss the feedback issued and to collaborate on any improvements to our solution we could make based upon the feedback. In our next face to face meeting with the customer we explained our architectural ideas and discussed the test application, we then went on to have a team meeting where we planned how to monitor the test application. We then sent the customer a series of emails detailing implementation and architectural decisions, including the qualities and views we identified, and also an example of a reporting engine. The customer replied with helpful feedback including additional qualities they felt were important to the system and positive feedback on the module view diagram.


\vfill
\bibliography{report-refs}
\bibliographystyle{IEEEtran}
\end{document}
