\section{Requirements Refinement}
\label{sec:requirements}

In this section, we shall discuss the changes to the requirements
since the initial report, and the reasons for these.

\subsection{Functional Requirements}
\label{sec:requirements-functional}

\subsubsection{Sensing Data}
\label{sec:requirements-functional-sensing}
\begin{description}
  \item[\fr{1.1}] The system shall provide an API for data input.

  \item[\fr{1.2}] The data input API shall require an identifier for
  the system being monitored.

  \item[\fr{1.3}] The data input API shall require data to be
  timestampe.

  \item[\fr{1.4}] The data input API shall allow data input clients
  to send key/value store ``payload'', which will then be made
  available to the analysis engine.

  \item[\fr{1.5}] A collection of common components of sensors shall
  be provided.

  \item[\fr{1.6}] An example, working, data input client shall be
  available for users.

  \item[\fr{2.1}]Data input clients shall timestamp data before
  sending it to the system.

  \item[\fr{2.2}] Timestamps shall be taken immediately when the
  data is gathered.
\end{description}

\fr{1.1}, \fr{1.2}, \fr{1.3}, and \fr{1.4} expand upon \fr{1} by
specifying how the clients will send data to the system, and what sort
of data is required. In our implementation of a data input client for
the test application, we found this to be a flexible and easy-to-use
architecture for gathering data. \fr{1.5} and \fr{1.6} fulfil the
requirement of tailorability of input sources, by providing users with
a way to write their own data input client easily, if they so choose.

\fr{2} requires that data be timestamped, but doesn’t specify which
part of the system does this. \fr{2.1} and \fr{2.2} assign this
responsibility to the data input clients, and mandate when timestamps
should reflect.

\subsubsection{Storing Data}
\label{sec:requirements-functional-storing}

\begin{description}
 \item[\fr{3.1}] The system shall access and modify data through a
 database accessor so that the database used can easily be changed
 according to need.

 \item[\fr{4.1}] The solution shall store per-system configuration in
 a file.

 \item[\fr{4.2}] The system shall allow authorised users to modify
 configuration. 
\end{description}

\fr{3.1} refines \fr{3} by specifying how the data is to be stored and
later accessed in a way which allows the system to cope with different
demands. Allowing the database to be changed without needing to
rewrite the rest of the system means that the system could just as
easily run in a very space-constrained environment, or with highly
structured data, as in a space-abundant environment, or with
loosely-structured data.

\fr{4.1} and \fr{4.2} allow the functionality in \fr{4}, \fr{6},
\fr{8}, \fr{9}, and \fr{10} to be implemented, by specifying how these
settings are to be defined and stored.

\subsubsection{Analysing Data}
\label{sec:requirements-functional-analysing}

\begin{description}
 \item[\fr{11.1}] The system shall provide a data analysis API.

 \item[\fr{11.2}] The system shall provide a simple analysis engine
 in the form of a rules engine.

 \item[\fr{11.3}] The data input API shall notify the analysis
 system when new data is received.

 \item[\fr{11.4}] The database accessor shall allow the analysis
 system to request the last several records.

 \item[\fr{11.5}] The database accessor shall allow the analysis
 system to request records between two timestamps.

 \item[\fr{12.1}] The system shall allow the user to define their
 own events.

 \item[\fr{12.2}] The system shall allow the user to define when any
 particular event is triggered.
\end{description}

\fr{11.2} is essentially the content of the low-fidelity prototype of
the admin centre that we provided to the customer, and refines \fr{11}
by providing a simple way for customers to detect events. For
nontrivial analysis, however, the customer requires the ability to use
their own analysis engines, and it is for this that \fr{11.1}
provides. Naturally, \fr{11.2} would be implemented on top of
this. \fr{11.3} allows analysis to happen on demand, without a need
for the analysis engine to needlessly poll the database asking for new
data, and \fr{11.4} and \fr{11.5} constrain the database accessor by
specifying how the analysis engine can retrieve data from it.

\fr{12.1} and \fr{12.2} refine \fr{12} by specifying that the user can
control events down to the level of creating their own, and having
total control over where events are triggered.

\subsubsection{Reporting Data}
\label{sec:requirements-functional-feedback}

\begin{description}
 \item[\fr{20}] The system shall allow analysis results to be sent
 to the data input API.

 \item[\fr{21}] The system shall allow for reports to change the way
 in which data is sensed.

 \item[\fr{22}] The system shall allow notifications to be sent back
 to the system being monitored.

 \item[\fr{23}] The admin centre of the system shall allow users to
 specify how notifications are handled. 
\end{description}

These requirements do not appear in the prior report at all, and were
discovered to be necessary following feedback from the customer
regarding the ability of the system to feed back the results of
analysis into itself. It was felt that our architecture did not make
it clear whether this was possible or not, and so we produced these
requirements, in consultation with the customer, to ensure clarity and
to remove any potential misunderstandings.
