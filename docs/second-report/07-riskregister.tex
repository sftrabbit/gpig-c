\section{Risk register}
\label{sec:riskregister}

A risk assessment has been performed as part of the interim
report to ensure the team is aware of any problems which could later
arise, and to provide a guide as to how to react when such problems
occur. The hazards (risks) have been identified by examining the 
current literature \cite{boehm1991software,jones1998minimizing}, drawing upon team members' past experiences and group 
discussion. Classification of the identified hazards, their impact on the system and their probability of occurrence is performed 
in accordance with the Risk Management Guide for Information Technology 
Systems \cite{stoneburner2002risk}. The areas these hazards impact were then analysed, as 
well as the probability of occurrence. These are then weighted so 
that we can identify the risks which are likely to 
have the greatest detrimental effect on the project. The likelihood 
score (LS), impact score (IS) and risk matrix score (RS) are listed 
in the table below.

\begin{longtable}[H]{| p{0.65cm} | p{2cm} | p{0.3cm} | p{0.3cm} | p{2.4cm} | p{3cm} | p{2.7cm} | p{0.4cm} |}
  \hline
  \cellcolor{titleColor}\textbf{Risk ID} &
  \cellcolor{titleColor}\textbf{Risk} &
  \cellcolor{titleColor}\textbf{LS} &
  \cellcolor{titleColor}\textbf{IS} &
  \cellcolor{titleColor}\textbf{Impact Description} &
  \cellcolor{titleColor}\textbf{Mitigation} &
  \cellcolor{titleColor}\textbf{Contingency} &
  \cellcolor{titleColor}\textbf{RS}\\
  
  \hline \textbf{R.1}
  & Loss of team member(s)
  & 6
  & 3
  & Internal and/or deadline failure
  & Ensure that no work relating to the project is outside of team
  version control; use of scrum methodology proactively aids work
  reallocation, ensuring team members are aware of all assigned work
  & Reallocation of work across remaining team members, may have to notify customer and possible deadline extension
  & 18 \\
  
  \hline \textbf{R.2}
  & Personnel under performance
  & 6
  & 3
  & Internal and/or deadline failure
  & Team strengths and weaknesses identified and work assigned appropriately, slack built into the schedule to account for under performance
  & Reallocation of work across remaining team members, may have to notify customer and possible deadline extension
  & 18 \\
  
  \hline \textbf{R.3}
  & Time pressures limiting amount of development time
  & 6
  & 4
  & Missing deadline, not fulfilling requirements, low quality deliverables
  & Define scope of work early, commit to complete requirements and allocate work to sprints 
  & Reduce functionality required and/or notify customer of later delivery
  & \\
  
  \hline \textbf{R.4}
  & Developing user interface that does not satisfy users requirements
  & 3
  & 3
  & Poor user experience, loss of confidence in the solution
  & Continuous, iterative development process, regularly incorporating users feedback on interface
  & Redevelop user interface
  & \\
  
  \hline \textbf{R.5}
  & Adding more functionality than necessary
  & 4
  & 3
  & Additional, unnecessary functionality, possible deadline miss
  & Ensure a clear, shared vision of the prototype system and collective ownership with open communication
  & \hl{???}
  & \\
  
  \hline \textbf{R.6}
  & Deprecation of required Google App Engine functionality
  & 3
  & 4
  & Unable to deliver required functionality
  & Research pending deprecations to ensure required functionality will be supported
  & Move to another key-store database (relatively simple)
  & \\
  
  
  \hline \textbf{R.7}
  & Commun-ications link to Google App Engine fails
  & 3
  & 4
  & Loss of service to users, loss of confidence in the system
  & A final commercial version would use the premium Google service which guarantees an uptime of at least 99.95\% (\nfrit10)
  & Additional data abstraction layer could be built to interface with another database service
  & \\
  
  \hline \textbf{R.8}
  & Data interception between our application and Google App Engine 
  & 3
  & 2
  & Loss of confidence in the system
  & A final system would encrypt traffic between our system and Google App Engine, currently just sending (non-sensitive) test data
  & Additional penetration testing to identify security flaws
  & \\
  
  \hline \textbf{R.9}
  & Data write/read times to/from Google App Engine too slow
  & 2
  & 5
  & Create a task backlog: gradual increase in load until system
  is non-operational (failure to meet capacity requirement \nfrit9)
  & Load testing of system throughout development to ensure \nfrit9
  & Re-engineering of system, possible need to switch datastore
  & \\
  
  \hline \textbf{R.10}
  & Implemen-tation does not facilitate data storage for a variety of systems
  & 3
  & 5
  & Failure to meet key modifiability requirements 
  & Adoption of a key-value database allows us the flexibility to store a variety of data
  & Examination of data we are struggling to store and implementation of additional data gateway for it
  & \\  
  
  \hline \textbf{R.11}
  & System performance required unattainable
  & 2
  & 4
  & Lead to failure to meet performance requirements 
  & Choice of implementation or choice of utilised hardware must offer
  enough in the way of performance to facilitate acceptable system
  performance.
  & All of, or combination of: optimisation of code, use of faster
  hardware, use of faster networks
  & 8 \\    
  
  \hline \textbf{R.12}
  & Included analysis components computing incorrect values
  & 2
  & 5
  & Failing to meet fundamental requirements, customer loses confidence in the system
  & Adopting a test driven approach to development throughout the development process
  & Re-engineering of failing components
  & \\
  
  \hline \textbf{R.13}
  & Analysis component computation too slow
  & 3
  & 4
  & Creates an analysis task backlog: gradual increase in load until system
  is non-operational
  & Load testing of analysis component throughout development, code optimisations 
  & Re-engineering of analysis components
  & \\  
  
  \hline \textbf{R.14}
  & Notification computation too slow
  & 2
  & 4
  & Creates a notification backlog: gradual increase in load until system
  is non-operational
  & Load testing of notification component throughout development, code optimisations 
  & Re-engineering of notification system
  & \\
  
  \hline \textbf{R.15}
  & GitHub becomes unavailable for a prolonged period of time
  & 2
  & 2
  & Difficulty merging work from different developers' machines
  & Distributed nature of Git means little work would be lost
  & Host our own git server within the Computer Science Department
  & \\ 
  
  \hline \textbf{R.16}
  & Malicious destruction of code repository on GitHub
  & 2
  & 2
  & Little impact as most developers computers would hold a near full copy of the central repository
  & Two-step authentication used on GitHub, distributed nature of Git means little work would be lost
  & Host our own git server within the Computer Science Department
  & \\    
    
  \hline
\end{longtable}       

The likelihood score defines probability of something occurring. Utilising
\textit{Kents Words of Estimative Probability}\cite{kent1966strategic}, with
`certain' weighted $7$ and `impossible' weighted $1$.

\begin{longtable}[H]{|c|c|l|c|c|c|c|c|}
  \cline{4-8} \multicolumn{3}{c|}{} & \multicolumn{5}{ c| }{Impact Score (Least$\rightarrow$Most)} \\
  \cline{4-8} \multicolumn{3}{c|}{} & 1 & 2 & 3 & 4 & 5 \\
  \cline{4-8} \multicolumn{3}{c|}{} & Negligible& Minor & Moderate& Major & Catastrophic \\

  \hline \multirow{7}{*}{Likelihood Score} & 7 & Certain & 7 & 14 & 21 & 28 & 35 \\

  \cline{2-8} & 6 & Almost certain & 6 & 12 & 18 & 24 & 30 \\
  \cline{2-8} & 5 & Probable& 5 & 10 & 15 & 20 & 25 \\
  \cline{2-8} & 4 & Chances about even & 4 & 8 & 12 & 16 & 20 \\
  \cline{2-8} & 3 & Probably not & 3 & 6 & 9 & 12 & 15 \\
  \cline{2-8} & 2 & Almost certainly not & 2 & 4 & 6 & 8 & 10 \\
  \cline{2-8} & 1 & Impossible & 1 & 2 & 3 & 4 & 5 \\
  \hline
\end{longtable}

\begin{longtable}[H]{ | p{2cm} | p{4cm} | p{8.5cm} | }
  \hline
  \cellcolor{titleColor}\textbf{Score} &
  \cellcolor{titleColor}\textbf{Risk Level} &
  \cellcolor{titleColor}\textbf{Recommended Response} \\

  \hline \textbf{23-35} & HIGH & Mitigation plan is
  required. Immediate action is required.\\

  \hline \textbf{11-22} & MEDIUM & To be included in the action plan
  and reviewed.\\

  \hline \textbf{0-10}& LOW & Included in action plan in limited
  scope. Minimum review. \\
  \hline
\end{longtable}
