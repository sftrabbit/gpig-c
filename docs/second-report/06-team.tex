\section{Team Structure}
\label{sec:team}

As in the initial report, we use the following acronyms for team members:

\begin{tabular}{ p{4cm} p{4cm} p{4cm} p{4cm} }
  \textbf{AJF}-Adam Fahie &
  \textbf{AIF}-Andrew Fairbairn &
  \textbf{TF}-Anthony Free &
  \textbf{TD}-Tom Davies \\
  \textbf{JM}-Joseph Mansfield &
  \textbf{RT}-Rosy Tucker &
  \textbf{MW}-Michael Walker & \\
\end{tabular}

Before beginning the second stage, we ascertained which modules everyone had in
the second half of the Autumn and the first half of the Spring terms, which we
then considered as we assigned tasks to people.

\subsection{Strengths and Weaknesses}
\label{sec:team-strengths}

To aid in deciding which technologies to use in the design and implementation
of the solution, we established what the strengths and weaknesses of each team
member were,

\begin{longtable}[H]{|p{2cm}|p{8cm}|p{5cm}|}
  \hline \cellcolor{titleColor}\textbf{Initials} &
  \cellcolor{titleColor}\textbf{Experience} &
  \cellcolor{titleColor}\textbf{Weaknesses}\\

  \hline AJF
  & Java, C, Ruby, Spring, Hibernate, Gradle, Design Patterns, Tomcat, JBoss,
  Git, SQL
  & Web Development, NoSQL, UI/UX, Technical Writing \\

  \hline AIF
  & Java, C++, Python, Javascript, PHP, UI/UX, Web Development, NoSQL, SQL
  & Spring, Technical Writing \\

  \hline TF
  & PHP, Javascript, Java, VB, C, Linux, MDE Developer, Git, SQL
  & UI/UX, Technical Writing \\

  \hline TD
  & Java, C, Python, Scala, Android, Swing, Machine Learning, SQL
  & NoSQL, Web Development, Git \\

  \hline JM
  & C, C++, Java, Python, UI/UX, Android, Web Development, Git, SQL
  & Hardware, NoSQL \\

  \hline RT
  & Java, C, Objective C, Javascript, Android, iOS, Swing, Design Patterns,
  Google AppEngine, HCI, NoSQL, SQL
  & Spring, Git \\

  \hline MW
  & C, Java, Python, Git
  & UX/UX, Technical Writing, Concurrency \\
  \hline
\end{longtable}

As can be clearly seen, everyone in the team is proficient with Java, and thus
it was a natural implementation choice for the prototype.

\subsection{Developmental Subteams}
\label{sec:team-subteams}

For the development of the prototype we split into three subteams,
concentrating on the different areas of the system. We then collaborated to
produce the reporting backend.

\begin{description}
  \item[Team Sense] implemented the data emitter for the test application,
    by producing reusable monitoring components. They also implemented the data
    input API as a multithreaded server, in order to meet requirements
    regarding simultaneous usage. (JM and MW)

  \item[Team Store] implemented the database abstraction interface, and also a
    concrete implementation using Google Appengine as a backing store. The API
    was kept generic in order to meet requirements regarding flexibility of
    database implementation. (TD and RT)

  \item[Team Analyse] implemented the analysis controller, and the analysis
    engine API. The engine API was implemented to be very sparse so that a large
    number of different concrete implementations could be supplied. (AJF, AIF,
    and TF)
\end{description}

