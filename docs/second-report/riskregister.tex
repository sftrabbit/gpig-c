\section{Risk register}
\label{sec:riskregister}

A risk assessment has been performed, based on the design decisions made throughout the initial and interim phases of the project, ensuring team members are aware how to react to problems when they occur. After feedback from the customer we took extra time to ensure both technical and procedural risks were identified early on. The risks from the Initial Report were reviewed and updated to reflect the more technical aspects of implementation. Risks relating to specific components were added, and some existing team planning risks were merged.

The hazards (risks) have been identified by examining the 
current literature \cite{boehm1991software,jones1998minimizing}, drawing upon team members' past experiences and group discussion. Classification of the identified hazards, their impact on the system and their probability of occurrence is performed in accordance with the Risk Management Guide for Information Technology Systems \cite{stoneburner2002risk}. The areas these hazards impact were then analysed, as well as the probability of occurrence. These are then weighted so that we can identify the risks which are likely to have the greatest detrimental effect on the project. The likelihood score (LS), impact score (IS) and risk matrix score (RS) are listed in the table below.

\begin{longtable}[H]{| p{0.6cm} | p{2.2cm} | p{0.26cm} | p{0.26cm} | p{2.7cm} | p{3cm} | p{2.6cm} | p{0.4cm} |}
  \hline
  \cellcolor{titleColor}\textbf{Risk ID} &
  \cellcolor{titleColor}\textbf{Risk} &
  \cellcolor{titleColor}\textbf{LS} &
  \cellcolor{titleColor}\textbf{IS} &
  \cellcolor{titleColor}\textbf{Impact Description} &
  \cellcolor{titleColor}\textbf{Mitigation} &
  \cellcolor{titleColor}\textbf{Contingency} &
  \cellcolor{titleColor}\textbf{RS}\\
  
  \hline \textbf{R.1}
  & Loss of team member(s).
  & 6
  & 3
  & Internal/external deadline failure.
  & Ensure all team project work is under version control.
 
  Use scrum methodology to proactively aids work reallocation, ensuring team members are aware of all assigned work.
  & Reallocate work across remaining team members, possibly notifying the customer and obtaining a deadline extension.
  & 18 \\
  
  \hline \textbf{R.2}
  & Team member(s) under performance.
  & 6
  & 3
  & Internal/external deadline failure.
  & Identify team member strengths and weaknesses and assign work accordingly. 
  
  Build slack into the schedule to allow for under performance.
  & Reallocate work across remaining team members, possibly notifying the customer and obtaining a deadline extension.
  & 18 \\
  
  \hline \textbf{R.3}
  & Time pressures limiting amount of development time.
  & 6
  & 4
  & Internal/external deadline failure.
  
  Not fulfilling requirements.
  
  Low quality deliverables.
  & Define scope of interim prototype early, commit to fulfilling requirements and allocate work to sprints.
  & Reduce functionality required and/or notify customer of later delivery.
  & 24 \\
  
  \hline \textbf{R.4}
  & Developed admin centre does not satisfy requirements.
  & 3
  & 3
  & Poor user experience.
  
  Loss of confidence in HUMS.
  & Continuous, iterative development process, regularly incorporating users feedback on the admin centre.
  & Redevelop admin centre.
  & 9\\
  
  \hline \textbf{R.5}
  & \hl{Deprecation of required Google App Engine functionality.} % Like R10
  & 3
  & 4

  & Unable to deliver required functionality.
  
  Loss of confidence in HUMS.
  & Research pending deprecations to ensure required functionality will be supported.
  & Migrate to another key-store database.
  & 12\\
  
  \hline \textbf{R.6}
  & Communic-ations link to Google App Engine fails.
  & 3
  & 4
  & Loss of service to users.

  Loss of confidence in the HUMS.
  & A final commercial version would use the premium Google service which guarantees an uptime of at least 99.95\% (\nfrit{9})
  & Additional data abstraction layer could be built to interface with another database service such as Heroku.
  & 12\\
  
  \hline \textbf{R.7}
  & Data interception between our application and Google App Engine.
  & 3
  & 2
  & Loss of confidence in the HUMS.
  & A final system would encrypt traffic between the HUMS and Google App Engine, currently just sending (non-sensitive) test data.
  & Additional penetration testing to identify security flaws.
  & 6\\
  
  \hline \textbf{R.8}
  & Data write/read times to and from Google App Engine are too slow.
  & 2
  & 5
  & Create a task backlog: gradual increase in load until system
  is non-operational.
  & Load test the HUMS throughout development.
  & Re-engineer HUMS, possibly switching datastore.
  & 10\\
  
  \hline \textbf{R.9}
  & HUMS does not facilitate data storage for a variety of systems.
  & 3
  & 5
  & Failure to meet key modifiability and flexibility requirements.
  & Adopt a key-value datastore allowing the flexibility to store a variety of data.
  & Examine difficult data and implement an additional SystemDataGateway,
  interfacing with a datastore capable of handling the data.
  & 15\\  
  
  \hline \textbf{R.10}
  & \hl{Deprecation of required Google App Engine functionality.} % Like R5
  & 3
  & 4
  & Required HUMS performance is unattainable.
  
  Unable to deliver required functionality.
  & Research pending deprecations to ensure required functionality will be supported.
  & Move to another key-store database (relatively simple).
  & 12\\
  
  \hline \textbf{R.11}
  & Included analysis components computing incorrect values.
  & 2
  & 5
  & Failing to meet fundamental requirements.
  
  Loss of confidence in the HUMS.
  & Adopt a test driven approach to development throughout the process
  ensuring expectations are met.
  & Re-engineer failing components.
  & 10\\
  
  \hline \textbf{R.12}
  & Included Analysis component computation too slow.
  & 3
  & 4
  & Creates an analysis task backlog: gradual increase in load until system
  is non-operational.
  & Load test analysis components throughout development.
  
    Optimise code.
  & Re-engineer analysis components.
  & 12\\  
  
  \hline \textbf{R.13}
  & Notification computation too slow.
  & 2
  & 4
  & Creates a notification backlog: gradual increase in load until system
  is non-operational.
  & Load test notification components throughout development.
 
  Optimise code.
  & Re-engineer notification system.
  & 8\\
  
  \hline \textbf{R.14}
  & GitHub becomes unavailable for a prolonged period of time.
  & 2
  & 2
  & Difficulty merging work from different machines.
  & Distributed nature of Git means little work would be lost.
  & Host our own Git server within the Computer Science Department.
  & 4\\ 
    \hline
\end{longtable}       

The likelihood score defines probability of something occurring. Utilising
\textit{Kents Words of Estimative Probability}\cite{kent1966strategic}, with
`certain' weighted $7$ and `impossible' weighted $1$.

\begin{tabular}[H]{|p{1.45cm}|c|l|c|c|c|c|c|}
  \cline{4-8} \multicolumn{3}{c|}{} & \multicolumn{5}{ c| }{Impact Score (Least$\rightarrow$Most)} \\
  \cline{4-8} \multicolumn{3}{c|}{} & 1 & 2 & 3 & 4 & 5 \\
  \cline{4-8} \multicolumn{3}{c|}{} & Negligible& Minor & Moderate& Major & Catastrophic \\

  \hline \multirow{7}{*}{Likelihood} & 7 & Certain & 7 & 14 & 21 & 28 & 35 \\

  \cline{2-8} & 6 & Almost certain & 6 & 12 & 18 & 24 & 30 \\
  \cline{2-8} & 5 & Probable& 5 & 10 & 15 & 20 & 25 \\
  \cline{2-8} & 4 & Chances about even & 4 & 8 & 12 & 16 & 20 \\
  \cline{2-8} & 3 & Probably not & 3 & 6 & 9 & 12 & 15 \\
  \cline{2-8} & 2 & Almost certainly not & 2 & 4 & 6 & 8 & 10 \\
  \cline{2-8} & 1 & Impossible & 1 & 2 & 3 & 4 & 5 \\
  \hline
\end{tabular}

\begin{longtable}[H]{ | p{2cm} | p{3cm} | p{8.5cm} | }
  \hline
  \cellcolor{titleColor}\textbf{Score} &
  \cellcolor{titleColor}\textbf{Risk Level} &
  \cellcolor{titleColor}\textbf{Recommended Response} \\

  \hline \textbf{23-35} & HIGH & Mitigation plan is
  required. Immediate action is required.\\

  \hline \textbf{11-22} & MEDIUM & To be included in the action plan
  and reviewed.\\

  \hline \textbf{0-10}& LOW & Included in action plan in limited
  scope. Minimum review. \\
  \hline
\end{longtable}

