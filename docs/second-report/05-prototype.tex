\section{Interim Prototype Implementation}
\label{sec:prototype}

\subsection{Technologies}
\label{sec:prototype-technologies}

We chose to develop the core of the HUMS prototype in Java. Java can
be compiled to bytecode that can be executed directly in all
environments that support a Java Virtual Machine, including various
operating systems and the x86, x86-64 and ARM CPU architectures,
making it highly portable between both desktop, server and embedded
systems.

% REF: http://www.oracle.com/technetwork/java/javase/config-417990.html

Whilst there may exist deployment platforms that are not supported, at
this stage, we do not see this as a problem since the underlying
architecture of the system is transferable across languages.

All members of our team were familiar with the Java programming
language and thus no time was spent having to get the team up to speed
on a new language. We did not impose a language restriction when
developing the data emitter and various engines as the data being
passed between them and the core are language independent, meaning end
users are free, even with the prototype, to implement engines in any
language they see fit. For the prototype we used both Java and
JavaScript when creating the engines, languages team members are
familiar with, allowing us to show off the versatility of the system.

We chose to use Google App Engine to host the prototype's database
module, displaying how easily it can incorporate existing, popular
technologies as plugins. The prototype could easily be extended to
work with other database technologies simply by implementing an
interface within the data abstraction layer.

We created a prototype admin centre to show how the HUMS could be
managed remotely, viewing data and modifying settings online, through
a web browser, whilst running as a service. The admin centre was
developed using the Bootstrap front-end web framework, including HTML,
JavaScript and CSS, in order to create a prototype realistically
simulating an interface to the HUMS that is accessible on desktop
computers, laptops and mobile devices alike.

We used Git for version control, allowing the entire team to
contribute to the project simultaneously, whilst keeping a reversible
history of all alterations made by each person.

\subsection{Current Functionality}
\label{sec:prototype-functionality}

\subsection{Testing} 
\label{sec:prototype-testing}

We created and implemented a test plan that laid out our key ideas
with respect to verifying our that solution fulfilled its
requirements. With a view to tracing the evolution from requirements
to implementation to testing, we allocated tests IDs and references to
their associated implementation components and requirement IDs. We
defined our plan, where appropriate, in terms of each module within
the system and what specific, testable attributes that module must
have to fulfil its requirements. We also further deconstructed the
testing into various levels of abstraction, mapping the higher level
requirements to their lower-level requisite requirements: acceptance
testing, system testing, integration testing and unit
testing.

For the analysis engine, we created a unit test to check each aspect of the engine. We utilised mockito, which is a framework for testing which allows emulation of results from methods. The particular analysis engine which we tested was the mean engine. The aim of this engine, which is part of the pre-defined engine set for our HUMS system, is to check set of values has a mean which lies within some user defined bounds. Therefore, our test cases setup to test that the mean values were valid. This meant using a pre-defined set of values to ensure the mean which the system calculated matched the true mean. Our test cases also dealt with the checking of means in and outside of bounds, as well as the cases where invalid data or system IDs were sent into the analysis engine.

% Does it make sense to have system/acceptance testing within a
% module? Given we are using a plugin architecture, I am not sure. TD

% TODO We probably need something about overall system & acceptance
% testing. These are mainly the white/black/grey box tests in the
% first report, e.g. NRF.9, "The system must cope with up to 2000 data
% input re- quests per second per HUMS instance."

\begin{description}
  \item[Data Emitter]
  \item[Data Input Layer]
  \item[Data Abstraction Layer] Given the high throughput and
    importance of storage to the function of the HUMS as a whole,the
    data abstraction layer's key attributes are availability and
    performance. When considering integration testing, the data
    abstraction layer interfaces with a wide selection of modules,
    including the datastore, the data input layer, the analysis
    controller and the reports engine. To test these interfaces, we
    referred to requirements pertaining to the storage of data,
    \textbf{FR.3} to \textbf{FR.10}, for guidance.
    % Should this be talking about the future, since it's a "plan", or
    % is it ok take about testing the prototype retrospectively?

    % Maybe referring to requirements is too much like system testing?

    At the unit testing level, focus was on the comparably complex,
    well defined module-internal functions, such as the generation and
    parsing of the JSON serialisation of objects.
  \item[Analysis Controller]
  \item[Analysis Engine]
  \item[Notification Generator]
  \item[Notification Engine]
  \item[Reports Engine]
\end{description}

\subsection{Evaluation on Test Application}
\label{sec:prototype-evaluation}

