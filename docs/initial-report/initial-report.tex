\documentclass[10pt,a4paper]{article}

\usepackage[margin=1.5cm]{geometry}
\usepackage[UKenglish]{babel}
\usepackage{enumitem}
\usepackage{fancyhdr}
\usepackage{graphicx}
\usepackage{multirow}
\usepackage[table]{xcolor}
\usepackage{float}

\definecolor{titleColor}{RGB}{138,201,242}
\pagestyle{fancy}
\lhead{T Davies, A Fahie, A Fairbairn, A Free, J Mansfield, R Tucker, M Walker}
\chead{}
\rhead{GPIG-C}
\cfoot{\vspace{-1cm} \thepage}

\setlist{nolistsep} % Reduces lots of white space around lists

\renewcommand{\headrulewidth}{0.4pt} % Add rules below header

\begin{document}

\title{\vspace{-1cm}GPIG-C Initial Report}
\author{Word count: \textbf{??} according to \textsl{detex \$FILE $\vert$ wc -w}}
\date{Friday, 25th October}
\maketitle
\thispagestyle{fancy} % Make sure header and footer appear on the front page

\section{Introduction}
\subsection{Single statement of need}
This project aims to deliver a tailorable Health and Usage Monitoring System (HUMS), tailorable to multiple target domains. The target for the system is any consumer needing to collect, store, analyse and report data from one or more data input clients.
\subsection{Intended audience of this document}
The intended audience of this document are both the developers and the customer. The document is structured in a manner which will interest all concerned parties. Requirements will follow the introduction, followed by use cases, a review of possible solutions and a risk register.

\section{Use cases}
\noindent \textbf{ID:} UC1\\
\textbf{Name:} Monitoring a system.\\
\textbf{Context:} The organisation has developed a system, for which they require health and usage monitoring.\\
\textbf{Primary Actor:} The organisation's technical representative.\\
\textbf{Secondary Actors:} The health and usage monitoring system, the organisation's system.\\
\textbf{Precondition:}  The organisation has a system that requires monitoring. The organisation's system and environment must conform to the constraint requirements detailed in this document. The end user has access to the facilities required to install the HUMS. The end user has basic technical computing knowledge.\\
\textbf{Trigger:} The end user sets up an account and attempts to integrate their system.\\
\textbf{Main Success Scenario:} The end user successfully manages to integrate their system with the HUMS.\\
\textbf{Main Success Postcondition:} The HUMS is successfully integrated with the user's system, allowing data to be passed through the input and output interfaces.\\
\textbf{Exception Scenarios:}
\begin{itemize}
\item The end user fails to integrate their system with the HUMS because their system does not conform to the data input interface.
\item The end user fails to integrate their system with the HUMS because their system does not conform to the output interface.
\item The end user chooses to abort the process.
\end{itemize}
\textbf{Exception Postcondition:} If the end user's system could not be integrated with the HUMS, the user is provided with diagnostic information and technical support needed to solve any problems. If the end user chose to quit the process, they are presented with access to technical support.\\\\
\noindent \textbf{ID:} UC2\\
\textbf{Name:} Analysing collected data.\\
\textbf{Context:} The organisation wishes to define how their data should be analysed.\\
\textbf{Primary Actor:} The end user.\\
\textbf{Secondary Actors:} The HUMS.\\
\textbf{Precondition:}
\begin{itemize}
\item The user has set up an account.
\item The organisation's sensor system and the HUMS have been successfully integrated.
\end{itemize}
\textbf{Trigger:} The user decides how they want to analyse their data abstractly.\\
\textbf{Main Success Scenario:}
\begin{itemize}
\item The user creates a concrete implementation of their abstract analysis system and interfaces it with the HUMS, allowing them to analyse data as per their definition.
\item We implement an analysis system on the user's behalf. The HUMS then analyses data as as defined by the user.
\end{itemize}
\textbf{Main Success Postcondition:} User can analyse data stored in the HUMS.\\
\textbf{Exception Scenarios:} User's analysis system does not conform to the analysis interface.\\
\textbf{Exception Postcondition:} HUMS can still collect and store data.\\\\
\noindent \textbf{ID:} UC3\\
\textbf{Name:} Reporting and notifications\\
\textbf{Context:} The organisation's sensor system, analysis system and the HUMS have been successfully integrated, and they now wish to define how users should be notified of events or receive reports.\\
\textbf{Primary Actor:} End user\\
\textbf{Secondary Actors:} The HUMS.\\
\textbf{Precondition:}
\begin{itemize}
\item The user has set up an account
\item The organisation's sensor system, analysis system and the HUMS have been successfully integrated.
\end{itemize}
\textbf{Trigger:} The user decides on the format and communication method used to keep them informed of the state of the system.\\
\textbf{Main Success Scenario:}
\begin{itemize}
\item The user implements a notification and reporting system with correctly integrates with the HUMS to keep end users informed.
\item We implement a notification and reporting system on behalf of the customer, allowing them to receive updates about the state of the system they are monitoring.
\end{itemize}
\textbf{Main Success Postcondition:}
\begin{itemize}
\item The user is notified when the analysis system fires an event
\item The user can request reports
\end{itemize}
\textbf{Exception Scenarios:} User's notification and reporting system implementation does not conform to the notification and reporting interface and so cannot function correctly.\\
\textbf{Exception Postcondition:} HUMS can still collect, store and analyse data.

\section{Requirements}
For this project, requirements have been divided into three categories, functional, non-functional and constraint. The requirements have been engineered such that each falls into one of those three categories, this prevents them from becoming too complex. Each requirement has been structured as simply and concisely as possible and been worded in such a way as to avoid ambiguity. The requirements represent the problem to be solved and serve as a contract between us, as the development team, and the customer. All requirements have therefore been checked with the customer to ensure the system described is the system they envisioned.\\
The functional requirements detail what inputs, behaviour and outputs the system must provide. The non-functional requirements specify the qualities of the system as opposed to its behaviour. Constraint requirements are those that apply to the entire system, including any constraints on the environment the system can be used in and any timing constraints. In order to ensure all requirements are verifiable, appropriate testing procedures have been included.

\subsection{Functional Requirements}
\subsubsection{Sensing Data}
\begin{table}
    \begin{tabular}{lll}
    \textbf{ID}   & \textbf{Description}                                                              & \textbf{Verification}                                                                                                                                                                                                                                              \\ \hline
    \textbf{FR.1} & Data input clients shall be able to push correctly structured data to the system. & Unit Testing\\\begin{itemize}\\\item Attempt to send correctly structured data to the system. Assert the data is correctly received.\\\item Attempt to send incorrectly structured data to the system. Assert the data failed structure validation.\\\end{itemize} \\
    \textbf{FR.2} & The system shall allocate a timestamp to new data.                                & Unit Testing\\\begin{itemize}\\\item Assert data is timestamped.\\\item Assert there is a ’happens before’ relation between any pair of timestamps, such that timestamps reflect the order in which data arrives.\\\end{itemize}                                   \\
	\end{tabular}
\end{table}
\subsubsection{Storing Data}
\begin{table}
    \begin{tabular}{|l|l|l|}
        \hline
        \textbf{ID}    & \textbf{Description}                                                                                                                           & \textbf{Verification}                                                                                                                                                                                                                                                                    \\ \hline
        \textbf{FR.3}  & The system shall store correctly structured data.                                                                                              & Unit Testing\\\begin{itemize}\\\item Attempt to store correctly structured data in the system. Assert that the data can be retrieved.\\\item Attempt to store incorrectly structured data in the system. Assert that the data is rejected.\\\end{itemize}                                \\ \hline
        \textbf{FR.4}  & The system shall allow the consumer to define a low storage threshold.                                                                         & Black Box Testing\\\begin{itemize}\\\item Attempt to set a low storage threshold. Assert attempt is successful.\\\end{itemize}                                                                                                                                                           \\ \hline
        \textbf{FR.5}  & The system shall send a notification when the consumers defined low storage threshold is reached.                                              & Unit Testing\\\begin{itemize}\\\item Simulate reaching the low storage threshold. Assert that the method invoking a notification is called.\\\end{itemize}                                                                                                                               \\ \hline
        \textbf{FR.6}  & The system shall allow the consumer to set an expiry time on data.                                                                             & Grey Box Testing\\\begin{itemize}\\\item Attempt to set an expiry time on data. Add the data to the system. Assert the data has been stored with the expiry time.\\\end{itemize}                                                                                                         \\ \hline
        \textbf{FR.7}  & The system will delete data when its expiry time is reached.                                                                                   & Integration Testing\\\begin{itemize}\\\item Assert no stored data is older than its defined expiry time.\\\end{itemize}                                                                                                                                                                  \\ \hline
        \textbf{FR.8}  & The system must store no more data records than the consumers defined storage quota.                                                           & Integration Testing\\\begin{itemize}\\\item Assert the maximum number of data records held by the consumer is less than or equal to their defined quota.\\\end{itemize}                                                                                                                  \\ \hline
        \textbf{FR.9}  & The system shall allow the user to define that, upon reaching their defined data storage quota, new data is no longer added.                   & Unit Testing\\\begin{itemize}\\\item Simulate reaching the an arbitrary storage quota. Attempt to send more data to the system. Assert the send failed due to insufficient storage.\\\end{itemize}                                                                                       \\ \hline
    	\textbf{FR.10} & The system shall allow the user to define that, upon reaching their defined data storage limit, old data is deleted to make room for new data. & Unit Testing\\\begin{itemize}\\\item Simulate reaching the an arbitrary storage quota. Attempt to send more data to the system. Assert the send succeeded and the new data has been added to storage. Assert that the previous oldest record in storage has been deleted.\\\end{itemize} \\ \hline
	\end{tabular}
\end{table}


\subsection{Non-functional requirements}

\subsubsection{Maintainability}
\begin{longtable}[H]{|p{1.5cm}|p{4.5cm}|p{10.5cm}|}
    \hline
    \cellcolor{titleColor}\textbf{ID} & \cellcolor{titleColor}\textbf{Description} & \cellcolor{titleColor}\textbf{Verification} \\ \hline
    \textbf{NFR.1} & The system shall be able to receive hardware changes without loss of previously stored data. & Recovery Testing\begin{itemize}\item Take a snapshot of a live system. Change a piece of hardware on the system and then check that, after the change, the data is still consistent with the snapshot.\end{itemize} \\ \hline
\end{longtable}

\subsubsection{Accessibility}
\begin{longtable}[H]{|p{1.5cm}|p{4.5cm}|p{10.5cm}|}
    \hline
    \cellcolor{titleColor}\textbf{ID} & \cellcolor{titleColor}\textbf{Description} & \cellcolor{titleColor}\textbf{Verification} \\ \hline
    \textbf{NFR.2} & Users shall be provided with documentation detailing how to use the system. & Acceptance Testing\begin{itemize}\item We will require a receipt verifying the availability and standard of the documentation when allowing the user access to the system documentation.\end{itemize} \\ \hline
    \textbf{NFR.3} & The system, when running externally on servers, must be accessible to end users who are in multiple geographic locations. & Black Box Testing\begin{itemize}\item Assert the networked system can be accessed from multiple geographic locations.\end{itemize} \\ \hline
\end{longtable}

\subsubsection{Security}
\begin{longtable}[H]{|p{1.5cm}|p{4.5cm}|p{10.5cm}|}
    \hline
    \cellcolor{titleColor}\textbf{ID} & \cellcolor{titleColor}\textbf{Description} & \cellcolor{titleColor}\textbf{Verification} \\ \hline
    \textbf{NFR.4} & The system shall only accept data from an input client providing valid credentials. & Unit testing\begin{itemize}\item Check that system both rejects unauthorised data and successfully accepts data from a source with valid credentials.\end{itemize} \\ \hline
    \textbf{NFR.5} & The system shall store data according to the relevant industry security standards. & Acceptance testing\begin{itemize}\item Have system externally verified against relevant standard setting body guidelines.\end{itemize} \\ \hline
\end{longtable}

\subsubsection{Testability}
\begin{longtable}[H]{|p{1.5cm}|p{4.5cm}|p{10.5cm}|}
    \hline
    \cellcolor{titleColor}\textbf{ID} & \cellcolor{titleColor}\textbf{Description} & \cellcolor{titleColor}\textbf{Verification} \\ \hline
    \textbf{NFR.6} & The system shall be tested to ensure all requirements are met before deployment. & System Testing\begin{itemize}\item Make sure all tests of other requirements have passed.\end{itemize} \\ \hline
    \textbf{NFR.7} & The customer will complete acceptance testing before the system is deployed. & Acceptance Testing\begin{itemize}\item The customer will be required to sign off the project upon passing acceptance testing.\end{itemize} \\ \hline
\end{longtable}

\subsubsection{Scalability}
\begin{longtable}[H]{|p{1.5cm}|p{4.5cm}|p{10.5cm}|}
    \hline
    \cellcolor{titleColor}\textbf{ID} & \cellcolor{titleColor}\textbf{Description} & \cellcolor{titleColor}\textbf{Verification} \\ \hline
    \textbf{NFR.8} & The system must be able to support at least 5 output clients per HUMS instance. & Black box testing\begin{itemize}\item Set up a HUMS instance and add 5 output clients and verify each can be sent a report.\end{itemize} \\ \hline
    \textbf{NFR.9} & The system must cope with up to 2000 data input requests per second per HUMS instance. & Grey Box Testing\begin{itemize}\item Set up a HUMS instance and send it 2000 data valid input requests per second and verify that all data is stored in the system.\end{itemize} \\ \hline
\end{longtable}

\subsubsection{Reliability}
\begin{longtable}[H]{|p{1.5cm}|p{4.5cm}|p{10.5cm}|}
    \hline
    \cellcolor{titleColor}\textbf{ID} & \cellcolor{titleColor}\textbf{Description} & \cellcolor{titleColor}\textbf{Verification} \\ \hline
    \textbf{NFR.10} & The system must be available for no less than 99.9\% of each month. & Alpha testing\begin{itemize}\item Allow the system to be used as it would be in the real world for one month and check that it is available for the required amount of time.\end{itemize} \\ \hline
    \textbf{NFR.11} & Data must be backed up within 24 hours of having been made available to the system. & Integration testing\begin{itemize}\item Run the system for over 24 hours and verify that any data older than 24 hours exists in the back ups.\end{itemize}White Box Testing\begin{itemize}\item Verify there is a method in place to automatically back up data before it being 24 hours since having been made available to the system.\end{itemize} \\ \hline
    \textbf{NFR.12} & Timestamps applied by the system must be accurate to within 5ms of UTC. & Grey Box Testing\begin{itemize}\item Send the system the maximum number of supported data input requests per second and check that all timestamps are to within the given accuracy.\end{itemize} \\ \hline
    \textbf{NFR.13} & The system shall dispatch notifications within 5ms of the event being triggered. & Unit Testing\begin{itemize}\item Simulate an event. Assert that the notification is dispatched within the time period specified.\end{itemize} \\ \hline
\end{longtable}


\subsection{Constraint Requirements}
\subsubsection{System Environment}
\begin{longtable}[H]{| p{1.5cm} | p{15cm} |}
		\hline
		\cellcolor{titleColor}\textbf{ID}		&	\cellcolor{titleColor}\textbf{Description}	\\	\hline
		\textbf{CR.1}      &       A valid data schema defining the form of data that will be entered into the system will be provided.         \\ \hline
		\textbf{CR.2}      &       A valid data schema defining the form of data that will be stored the system will be provided.         \\ \hline
		\textbf{CR.3}      &       The system will be presented with valid credentials by the consumer’s data input clients.         \\ \hline
		\textbf{CR.4}      &       Valid output clients for the system will be specified by the consumer.         \\ \hline
		\textbf{CR.5}      &       The hardware that the system runs on will support the runtime of our software solution.         \\ \hline
		\textbf{CR.6}      &       The system will have access to the network to which the data sources are connected.         \\ \hline
\end{longtable}

\subsubsection{Time}
\begin{longtable}[H]{| p{1.5cm} | p{15cm} |}
		\hline
		\cellcolor{titleColor}\textbf{ID}		&	\cellcolor{titleColor}\textbf{Description}	\\	\hline
		\textbf{CR.7}      &       The customer will receive an initial project plan detailing the project requirements no later than October 25th 2013.         \\ \hline
		\textbf{CR.8}      &       The customer will receive an interim report detailing project progress no later February 14th 2014.         \\ \hline
		\textbf{CR.9}      &       The customer will receive a final report detailing the proposed system no later than May 28th 2014.         \\ \hline
		\textbf{CR.10}      &       The customer will be presented with a system prototype on May 30th 2014.         \\ \hline
\end{longtable}

\subsubsection{Development}
\begin{longtable}[H]{| p{1.5cm} | p{15cm} |}
		\hline
		\cellcolor{titleColor}\textbf{ID}		&	\cellcolor{titleColor}\textbf{Description}	\\	\hline
		\textbf{CR.11}      &       The development team will be comprised of seven software engineers.         \\ \hline
		\textbf{CR.12}      &       Any requirement changes outside of this document will be negotiated separately.         \\ \hline
\end{longtable}


\subsection{Constraint requirements}

\section{Possible solutions}

\subsection{Proposed solution}
Our proposed solution uses the concept of a client-server architecture, where the client is the consumer?s system, sensing the data, and the server is the system processing the data. These could exist within the same package or on the same device, but they could also be geographically-distant machines connected across the internet. The solution brings together the best elements from our possible solutions for each section of the system, utilising a plugin architecture for several components.
\\ \\
We will provide an input API for data input clients to push new data to the system, accepting any correctly formatted sensor data sent on the correct network (DD.1). Achieving correctly formatted data may require changes to the consumers system, or using middleware. This approach makes our system flexible, supporting a wide range of data input clients, meaning in future the solution could easily extend across domains.
\\ \\
For data storage, the only applicable solution in this scenario is a database (DD.2). Using polyglot persistence gives us the option of choosing a database technology most suited to the consumer?s needs, which we will determine during development. To make our choice we will look at all the relevant technologies available and evaluate strengths and weaknesses in relation to other design decisions for the data being stored, and how it will be accessed.
\\ \\
The data analysis part of the system will provide plugins offering analysis tools, as well as an API for consumers to extend and create their own specialised analysis rules (DD.4). Offering an API makes the system more tailorable to different consumers and domains, as generic analysis tools could not hope to meet every consumers needs by itself. Since some customers may not want to take on the task of creating a whole analysis system themselves, we will offer domain-specific plugins to simplify the process. This might include graphical tools or other user-friendly features. The range of plugins offered can be expanded as the system expands into more domains.
\\ \\
The system will include various output features for generating reports (DD.5) and providing notifications (DD.6). For reports, we will allow the consumer to customise the types of reports created. They will be able to request reports in a variety of formats, including human-friendly formats, e.g. PDF, and common data interchange formats. Consumers can use these data formats as a data API to build their own reports or for any other purpose. This will allow the consumer flexibility to produce what they want, whilst also offering them easy to use built-in tools.
\\ \\
Our system will provide plugins to support various types of notification that can be triggered based upon customisable events. Notifications will include  emails, alongside more generic notifications formatted in standard formats such as XML, just as with reports. The generic notifications could then be processed further by the consumer, allowing them to tailor how notifications are used and delivered.
\\ \\
The system will also provide an administration centre (DD.3) that lets the consumer configure any customisable system settings including, setting up events that trigger notifications and defining data schemas. This tool will be offered through an intuitive web based interface which is only accessible by the authorised consumer.
\\ \\
For the prototype we hope to demonstrate at the end of the project, we plan to implement most of the solution described above. The core of the system, including sensing, storing, analysing and reporting data, will be functional, as will some of the basic components built utilising the plugin architecture. Some additional simple components may be required when demonstrating, such as a tool to simulate data being input to the system. 

\subsubsection{Design Decision}
\begin{table}[h!]
	\begin{tabular}{ | p{3cm} | p{3.5cm} | p{9.5cm} | }
		\hline
			\textbf{Design Decision ID}&	\textbf{Requirements Fulfilled}&	\textbf{Description}	\\ \hline
			\textbf{DD.1}&	FR.1, FR.2&	Provide an input API which data input clients wishing to push data to the system must conform to.	\\ \hline
			\textbf{DD.2}&	FR.7, FR.8&	Use a database to store data.	\\ \hline
			\textbf{DD.3}&	FR.4, FR.5, FR.6 FR.9, FR.10, FR.14, FR.15, FR.18, FR.19&	The system will provide an admin centre which allows the consumer to define data schemas and any configurable system settings.	\\ \hline
			\textbf{DD.4}&	FR.11&	The system will provide an analysis API, which the consumer can choose to implement, it will also provide a set of plugins conforming to this API which perform commonly used analysis functions.	\\ \hline
			\textbf{DD.5}&	FR.17&	The system will provide a set of plugins which allow the consumer to pull reports from the system in a variety of common formats, including high level formats such as PDF and low level formats such as XML.	\\ \hline
			\textbf{DD.6}&	FR.13, FR.13, FR.16&	The system will provide a set of plugins which push notifications of different formats to end users or data output clients.	\\ \hline
	\end{tabular}
\end{table}


\begin{figure}[hptb]
  \centering
\includegraphics[width=0.75\textwidth]{system-architecture.pdf}
  \caption{System diagram}
\end{figure}

\section{Team organisation}

\subsection{Team members}
A list of team members is provided below, accompanied by a list of roles and assignments relevant for the course of the project. The responsibilities have been divided in a such a way that effort for all team members is equal and fairly distributed. Assignees have been given roles which, where possible, utilise their identified skills, and minimise exposure to established weaknesses. \\
\\
\begin{tabular}{  p{4cm}  p{4cm}  p{4cm} p{4cm}  }
\textbf{AJF}-Adam Fahie            &         \textbf{AIF}-Andrew Fairbairn    &     \textbf{TF}-Anthony Free          &        \textbf{TD}-Tom Davies \\
\textbf{JM}-Joseph Mansfield     &        \textbf{RT}-Rosy Tucker           &          \textbf{MW}-Michael Walker \\
\end{tabular}

\subsection{Roles}
\subsubsection{Administrator Roles} 
\begin{table}[H]
\begin{tabular}{ | p{3.5cm} | p{8.5cm} | p{4.5cm} |}
\hline
\textbf{Role}	        &   \textbf{Description}                                                                   & \textbf{Assignees}                         \\ \hline
Point of Contact         &   Buffer between customer and team.                                        & RT                                                 \\ \hline
Team Coordinator      &   Manage team communication and organise meetings.            & JM           				         \\ \hline
Minutes Secretary     &   Take minutes during meetings.                                                & AJF,AIF,TF,TD,JM,RT,MW             \\ \hline

\end{tabular}
\end{table}

\subsubsection{Technical Roles}
\begin{longtable}[H]{ | p{3.5cm} | p{8.5cm} | p{4.5cm} |}
\hline
\cellcolor{titleColor}\textbf{Role}	       		&   \cellcolor{titleColor}\textbf{Description}                                                                   											& \cellcolor{titleColor}\textbf{Assignees}                  \\ \hline
Editors         			&   Review and format draft documentation.                                        											& JM,TD,RT                               \\ \hline
Software Architect      	&   Responsible for ensuring requirements are met, and software is consistent.         								& AJF         			 	\\ \hline
Requirements Analysis     &   Identification of customer needs and requirements.                                              								& RT,AJF,JM,TD,AIF 		\\ \hline
Developer         		&   Developing software to agreed standards.                                      		 								&  AJF,AIF,TF,TD,JM,RT,MW     \\ \hline
UX/UI Designer     		&   Accessibility, UI/UX and user stories.							                								& JM,RT,AIF           			 \\ \hline
Security     			&   Advising on relevant security standards and protocols.                                           								& TF,MW				   	 \\ \hline
Penetration Tester       	&   Test HUMS for exploits and security weaknesses.                           	                 							& AJF,TF,AIF,TD,MW              	\\ \hline
Software Tester     		&   Creating or assisting in the creation of tests.          													& AJF,AIF,TF,TD,JM,RT,MW  	 \\ \hline
Risk Manager     		&   Enhancing chance of a successful delivery.                                               									& RT.TD				   	 \\ \hline
Database Admin		&   Design of database strategies and schemas.                            		                 							& AJF,AIF,MW                           	 \\ \hline
Legal    				&   Avoiding copyright license and patent violations.          					       								& JM           				 \\ \hline
Quality Assurance             &   Ensure cohesion between system modules, analysing system performance and enforce code quality standards.  		& JM,TD				    	\\ \hline
Product Owner        		&   Representing the views of the customer.                                       										 	& RT                                        	\\ \hline
Scrum Master      		&   Ensuring Scrum process is followed by all team members, and ensuring good inter-team communication.            		& RT           				\\ \hline
\end{longtable}

\subsection{Time and Task Management}

\section{Development plan}

\subsection{Software engineering methodology}
When developing the HUMS system, we chose to take an agile approach to development and team management. Our implementation will follow a plugin architecture, with one central codebase providing the core functionality and plugins which provide domain tailorability. If a non-agile approach was to be taken, a plugin architecture would be infeasible as those methodologies require all requirements and documentation to be completed before development commences. For example, the V-Model and waterfall model are linear processes, where requirements are only discussed and designed before implementation commences. This linearity gives no support for changes in requirements during the later phases. Given the time constraints and the need for the system to be able to evolve into various domains, these linear development methodologies are impractical. \\
Agile software development, however, allows the project to split into smaller subprojects called `iterations' \cite{hazzan2008agile}. Each iteration allows for a different section of the project to be planned, documented, developed and tested. It also allows for the development team to be split into smaller sub-teams working simultaneously on different areas on the project. At the end of each iteration project priorities can be re-evaluated and the team structure altered. 
\cite{hazzan2008agile} identifies three key perspectives of software development: human, organisational and technological. The scrum methodology harnesses two of those perspectives, providing a platform for both team and project organisation. Unfortunately scrum does not provide a clear approach to development from a technological perspective, providing no clear guidelines on how code should be created or structured. However, the test driven approach does provide a clear method of developing software from a technological perspective. This encourages the development of unit and acceptance tests before the development of code, the code itself is then written to make the tests pass. Using both methodologies side by side appears to be the best approach for this project, allowing for a well managed and organised team to produce well structured and reliable code. Feature driven development could be used in place of test driven and would provide a good platform for plugin generation, allowing each plugin to be designed and implemented independently. However, it conflicts with the project organisation aspects of the scrum methodology, meaning they could not be used together without creating confusion. Since the scrum method provides more guidance for managing team structure and monitoring project progress, feature driven development was discounted in favour of scrum.

\subsection{Schedule}


\section{Risk register}
%\begin{table}[H]
	\begin{tabular}{  p{2cm}  p{1.5cm}  p{3cm} | p{2cm} | p{2cm} | p{2cm} | p{2cm} | p{2cm} | }
		\cline{4-8}
		& &	&		\multicolumn{5}{ |c| }{Impact Score (Least->Most)}&					\\ \cline{4-8}
		& &	&	1&	2&	3&	4&	5&	\\ \cline{4-8}
		& &	&	Negligible&	Minor&	Moderate&	Major&	Catastrophic&	\\ \hline
		\multicolumn{1}{ |c| }{\multirow{7}{*}{Likelihood Score}} &	7&	Certain&	7&	14&	21&	28&	35&	\\ \hline
		\multicolumn{1}{ |c  }{} &		6&	Almost certain&	6&	12&	18&	24&	30&	\\ \hline
		\multicolumn{1}{ |c  }{} &		5&	Probable&	5&	10&	15&	20&	25&	\\ \hline
		\multicolumn{1}{ |c  }{} &		4&	Chances about even&	4&	8&	12&	16&	20&	\\ \hline
		\multicolumn{1}{ |c  }{} &		3&	Probably Not&	3&	6&	9&	12&	15&	\\ \hline
		\multicolumn{1}{ |c  }{} &		2&	Almost certainly not&	2&	4&	6&	8&	10&	\\ \hline
		\multicolumn{1}{ |c  }{} &		1&	Impossible&	1&	2&	3&	4&	5&	\\ \hline
	\end{tabular}
\end{table}
														


\section{Customer communication}


\end{document}
