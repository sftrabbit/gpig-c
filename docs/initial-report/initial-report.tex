\documentclass[10pt,a4paper]{article}

\usepackage[margin=1.5cm]{geometry}
\usepackage[UKenglish]{babel}
\usepackage{enumitem}
\usepackage{fancyhdr}
\usepackage{graphicx}
\usepackage{multirow}
\usepackage[table]{xcolor}
\usepackage{float}
\usepackage{longtable}
\usepackage{parskip}
\usepackage[small,compact]{titlesec} 

\definecolor{titleColor}{RGB}{138,201,242}
\pagestyle{fancy}
\lhead{T Davies, A Fahie, A Fairbairn, A Free, J Mansfield, R Tucker, M Walker}
\chead{}
\rhead{GPIG-C}
\cfoot{\vspace{-0.6cm} \thepage}

\setlist{nolistsep} % Reduces lots of white space around lists

\renewcommand{\headrulewidth}{0.4pt} % Add rules below header

\begin{document}

% Worst latex ever, but it looks so pretty!
\title{\vspace{-2cm}GPIG-C Initial Report\vspace{-0.7cm}}
\author{\fontsize{7}{8pt}Word count: \input{wc}\footnote{\textit{Using TeXCount, excluding all tables and figures except the Risk Register}}}
\date{\vspace{-0.6cm}\fontsize{7}{6pt}Friday, 25th October 2013}
\maketitle
\vspace{-0.4cm}
\hline
\thispagestyle{fancy} % Make sure header and footer appear on the front page

\section{Introduction} %40 words
This document is intended for both ourselves and the customer. It contains
an initial analysis of requirements and a development plan. A proposed solution,
based on identified use cases and requirements, is presented herein.
 
\subsection{Single statement of need} %40 words
We aim to deliver a Health and Usage Monitoring System (HUMS), tailorable to
multiple target domains. The target user for the system is any consumer needing
to collect, store, analyse and report data from one or more data input clients.

\section{Use cases}
Three use cases were identified after initial communications with the customer and used to generate the requirements. 

\begin{description}
	\item[ID] $UC1$
	\item[Name]  System Monitoring.
	\item[Context] The consumer develops a system, for which they
	               require a HUMS.
	\item[Primary Actor] The consumer's technical representative.
	\item[Secondary Actors] The HUMS, the consumer's system.
	\item[Preconditions] ~
		\begin{itemize}
			\item The consumer has a system that requires monitoring.
	                 \item The consumer's system and environment must conform to constraint
	                     requirements detailed in this document. 
	                 \item The end user has access to the facilities required to install
	                    the HUMS. The end user has basic computing knowledge.
	          \end{itemize}
	\item[Trigger] The technical representative sets up an account and attempts to integrate
	               their system.
	\item[Main Success Scenario] The end user successfully manages to integrate
			their system with the HUMS.
	\item[Main Success Postcondition] Data can be passed between the HUMS and the 
					consumers system.
	\item[Exception Scenarios] ~
			\begin{itemize}
				\item The end user fails to integrate their system because
				      their system does not conform to the data input interface.
				\item The end user fails to integrate their system because
				      their system does not conform to the output interface.
				\item The end user chooses to abort the process.
			\end{itemize}
	\item[Exception Postcondition]
			The user is provided with diagnostics and technical support. 
			If the end user chose to quit the process, they are
			presented with access to technical support.
\end{description}

\vspace{\baselineskip}
\begin{description}
	\item[ID] $UC2$
	\item[Name] Analysing collected data.
	\item[Context] The consumer wishes to define how their data should be
	               analysed.
	\item[Primary Actor] End user.
	\item[Secondary Actors] The HUMS.
	\item[Preconditions] ~
			\begin{itemize}
				\item The user has set up an account.
				\item The consumer's sensor system and the HUMS have been
				      successfully integrated.
			\end{itemize}
	\item[Trigger] The user decides how they want to analyse their data
	               abstractly.
	\item[Main Success Scenarios] ~
			\begin{itemize}
				\item The user creates a concrete implementation of their abstract
				      analysis system and interfaces it with the HUMS, allowing them to
				      analyse data as per their definition.
			\end{itemize}
	\item[Main Success Postcondition] User can analyse data stored in the HUMS.
	\item[Exception Scenarios] User's analysis system does not conform to the
			analysis interface.
	\item[Exception Postcondition] HUMS can still collect and store data.
\end{description}

\vspace{\baselineskip}

\begin{description}
	\item[ID] $UC3$
	\item[Name] Reporting and notifications.
	\item[Context] The consumer's sensor system, analysis system and the HUMS
	               have been successfully integrated, and they now wish to define
	               how their users should be notified of events or receive reports.
	\item[Primary Actor] End user.
	\item[Secondary Actors] The HUMS.
	\item[Preconditions] ~
			\begin{itemize}
			\item The user has set up an account.
			\item The consumer's sensor system, analysis system and the HUMS have
			      been successfully integrated.
			\end{itemize}
	\item[Trigger] The user decides on the format and communication method used
	               to keep them informed of the state of the system.
	\item[Main Success Scenarios] ~
			\begin{itemize}
				\item The user implements a notification and reporting system which
				      correctly integrates with the HUMS.
				\item The user uses a pre-made notification and reporting system plugin,
				 allowing them to receive updates about the state of the
				      system they are monitoring.
			\end{itemize}
	\item[Main Success Postconditions] ~
			\begin{itemize}
				\item The user is notified when the analysis system fires an event.
				\item The user can request reports.
			\end{itemize}
	\item[Exception Scenarios] User's notification and reporting system
			implementation does not conform to the notification and reporting
			interface.
	\item[Exception Postcondition] HUMS can still collect, store and analyse data.
\end{description}

\section{Requirements}

For this project, requirements have been divided into three categories:
functional, non-functional and constraint. The requirements have been
engineered such that each falls into one of these three categories, which
prevents them from becoming too complex. Each requirement has been
structured as simply and concisely as possible and been worded in such a way
as to avoid ambiguity. The requirements represent the problem to be solved
and serve as a contract between us, as the development team, and the
customer. All requirements have therefore been checked with the customer to
ensure the system described is the system they envisioned.

The functional requirements detail what inputs, behaviour and outputs the system
must provide. The non-functional requirements specify the qualities of the
system as opposed to its behaviour. Constraint requirements are those that apply
to the entire system, including any constraints on the environment the system
can be used in and any timing constraints. In order to ensure all requirements
are verifiable, appropriate testing procedures have been included.

\subsection{Functional Requirements}
\subsubsection{Sensing Data}
\begin{table}
    \begin{tabular}{lll}
    \textbf{ID}   & \textbf{Description}                                                              & \textbf{Verification}                                                                                                                                                                                                                                              \\ \hline
    \textbf{FR.1} & Data input clients shall be able to push correctly structured data to the system. & Unit Testing\\\begin{itemize}\\\item Attempt to send correctly structured data to the system. Assert the data is correctly received.\\\item Attempt to send incorrectly structured data to the system. Assert the data failed structure validation.\\\end{itemize} \\
    \textbf{FR.2} & The system shall allocate a timestamp to new data.                                & Unit Testing\\\begin{itemize}\\\item Assert data is timestamped.\\\item Assert there is a ’happens before’ relation between any pair of timestamps, such that timestamps reflect the order in which data arrives.\\\end{itemize}                                   \\
	\end{tabular}
\end{table}
\subsubsection{Storing Data}
\begin{table}
    \begin{tabular}{|l|l|l|}
        \hline
        \textbf{ID}    & \textbf{Description}                                                                                                                           & \textbf{Verification}                                                                                                                                                                                                                                                                    \\ \hline
        \textbf{FR.3}  & The system shall store correctly structured data.                                                                                              & Unit Testing\\\begin{itemize}\\\item Attempt to store correctly structured data in the system. Assert that the data can be retrieved.\\\item Attempt to store incorrectly structured data in the system. Assert that the data is rejected.\\\end{itemize}                                \\ \hline
        \textbf{FR.4}  & The system shall allow the consumer to define a low storage threshold.                                                                         & Black Box Testing\\\begin{itemize}\\\item Attempt to set a low storage threshold. Assert attempt is successful.\\\end{itemize}                                                                                                                                                           \\ \hline
        \textbf{FR.5}  & The system shall send a notification when the consumers defined low storage threshold is reached.                                              & Unit Testing\\\begin{itemize}\\\item Simulate reaching the low storage threshold. Assert that the method invoking a notification is called.\\\end{itemize}                                                                                                                               \\ \hline
        \textbf{FR.6}  & The system shall allow the consumer to set an expiry time on data.                                                                             & Grey Box Testing\\\begin{itemize}\\\item Attempt to set an expiry time on data. Add the data to the system. Assert the data has been stored with the expiry time.\\\end{itemize}                                                                                                         \\ \hline
        \textbf{FR.7}  & The system will delete data when its expiry time is reached.                                                                                   & Integration Testing\\\begin{itemize}\\\item Assert no stored data is older than its defined expiry time.\\\end{itemize}                                                                                                                                                                  \\ \hline
        \textbf{FR.8}  & The system must store no more data records than the consumers defined storage quota.                                                           & Integration Testing\\\begin{itemize}\\\item Assert the maximum number of data records held by the consumer is less than or equal to their defined quota.\\\end{itemize}                                                                                                                  \\ \hline
        \textbf{FR.9}  & The system shall allow the user to define that, upon reaching their defined data storage quota, new data is no longer added.                   & Unit Testing\\\begin{itemize}\\\item Simulate reaching the an arbitrary storage quota. Attempt to send more data to the system. Assert the send failed due to insufficient storage.\\\end{itemize}                                                                                       \\ \hline
    	\textbf{FR.10} & The system shall allow the user to define that, upon reaching their defined data storage limit, old data is deleted to make room for new data. & Unit Testing\\\begin{itemize}\\\item Simulate reaching the an arbitrary storage quota. Attempt to send more data to the system. Assert the send succeeded and the new data has been added to storage. Assert that the previous oldest record in storage has been deleted.\\\end{itemize} \\ \hline
	\end{tabular}
\end{table}


\subsection{Non-functional requirements}

\subsubsection{Maintainability}
\begin{longtable}[H]{|p{1.5cm}|p{4.5cm}|p{10.5cm}|}
    \hline
    \cellcolor{titleColor}\textbf{ID} & \cellcolor{titleColor}\textbf{Description} & \cellcolor{titleColor}\textbf{Verification} \\ \hline
    \textbf{NFR.1} & The system shall be able to receive hardware changes without loss of previously stored data. & Recovery Testing\begin{itemize}\item Take a snapshot of a live system. Change a piece of hardware on the system and then check that, after the change, the data is still consistent with the snapshot.\end{itemize} \\ \hline
\end{longtable}

\subsubsection{Accessibility}
\begin{longtable}[H]{|p{1.5cm}|p{4.5cm}|p{10.5cm}|}
    \hline
    \cellcolor{titleColor}\textbf{ID} & \cellcolor{titleColor}\textbf{Description} & \cellcolor{titleColor}\textbf{Verification} \\ \hline
    \textbf{NFR.2} & Users shall be provided with documentation detailing how to use the system. & Acceptance Testing\begin{itemize}\item We will require a receipt verifying the availability and standard of the documentation when allowing the user access to the system documentation.\end{itemize} \\ \hline
    \textbf{NFR.3} & The system, when running externally on servers, must be accessible to end users who are in multiple geographic locations. & Black Box Testing\begin{itemize}\item Assert the networked system can be accessed from multiple geographic locations.\end{itemize} \\ \hline
\end{longtable}

\subsubsection{Security}
\begin{longtable}[H]{|p{1.5cm}|p{4.5cm}|p{10.5cm}|}
    \hline
    \cellcolor{titleColor}\textbf{ID} & \cellcolor{titleColor}\textbf{Description} & \cellcolor{titleColor}\textbf{Verification} \\ \hline
    \textbf{NFR.4} & The system shall only accept data from an input client providing valid credentials. & Unit testing\begin{itemize}\item Check that system both rejects unauthorised data and successfully accepts data from a source with valid credentials.\end{itemize} \\ \hline
    \textbf{NFR.5} & The system shall store data according to the relevant industry security standards. & Acceptance testing\begin{itemize}\item Have system externally verified against relevant standard setting body guidelines.\end{itemize} \\ \hline
\end{longtable}

\subsubsection{Testability}
\begin{longtable}[H]{|p{1.5cm}|p{4.5cm}|p{10.5cm}|}
    \hline
    \cellcolor{titleColor}\textbf{ID} & \cellcolor{titleColor}\textbf{Description} & \cellcolor{titleColor}\textbf{Verification} \\ \hline
    \textbf{NFR.6} & The system shall be tested to ensure all requirements are met before deployment. & System Testing\begin{itemize}\item Make sure all tests of other requirements have passed.\end{itemize} \\ \hline
    \textbf{NFR.7} & The customer will complete acceptance testing before the system is deployed. & Acceptance Testing\begin{itemize}\item The customer will be required to sign off the project upon passing acceptance testing.\end{itemize} \\ \hline
\end{longtable}

\subsubsection{Scalability}
\begin{longtable}[H]{|p{1.5cm}|p{4.5cm}|p{10.5cm}|}
    \hline
    \cellcolor{titleColor}\textbf{ID} & \cellcolor{titleColor}\textbf{Description} & \cellcolor{titleColor}\textbf{Verification} \\ \hline
    \textbf{NFR.8} & The system must be able to support at least 5 output clients per HUMS instance. & Black box testing\begin{itemize}\item Set up a HUMS instance and add 5 output clients and verify each can be sent a report.\end{itemize} \\ \hline
    \textbf{NFR.9} & The system must cope with up to 2000 data input requests per second per HUMS instance. & Grey Box Testing\begin{itemize}\item Set up a HUMS instance and send it 2000 data valid input requests per second and verify that all data is stored in the system.\end{itemize} \\ \hline
\end{longtable}

\subsubsection{Reliability}
\begin{longtable}[H]{|p{1.5cm}|p{4.5cm}|p{10.5cm}|}
    \hline
    \cellcolor{titleColor}\textbf{ID} & \cellcolor{titleColor}\textbf{Description} & \cellcolor{titleColor}\textbf{Verification} \\ \hline
    \textbf{NFR.10} & The system must be available for no less than 99.9\% of each month. & Alpha testing\begin{itemize}\item Allow the system to be used as it would be in the real world for one month and check that it is available for the required amount of time.\end{itemize} \\ \hline
    \textbf{NFR.11} & Data must be backed up within 24 hours of having been made available to the system. & Integration testing\begin{itemize}\item Run the system for over 24 hours and verify that any data older than 24 hours exists in the back ups.\end{itemize}White Box Testing\begin{itemize}\item Verify there is a method in place to automatically back up data before it being 24 hours since having been made available to the system.\end{itemize} \\ \hline
    \textbf{NFR.12} & Timestamps applied by the system must be accurate to within 5ms of UTC. & Grey Box Testing\begin{itemize}\item Send the system the maximum number of supported data input requests per second and check that all timestamps are to within the given accuracy.\end{itemize} \\ \hline
    \textbf{NFR.13} & The system shall dispatch notifications within 5ms of the event being triggered. & Unit Testing\begin{itemize}\item Simulate an event. Assert that the notification is dispatched within the time period specified.\end{itemize} \\ \hline
\end{longtable}


\subsection{Constraint Requirements}
\subsubsection{System Environment}
\begin{longtable}[H]{| p{1.5cm} | p{15cm} |}
		\hline
		\cellcolor{titleColor}\textbf{ID}		&	\cellcolor{titleColor}\textbf{Description}	\\	\hline
		\textbf{CR.1}      &       A valid data schema defining the form of data that will be entered into the system will be provided.         \\ \hline
		\textbf{CR.2}      &       A valid data schema defining the form of data that will be stored the system will be provided.         \\ \hline
		\textbf{CR.3}      &       The system will be presented with valid credentials by the consumer’s data input clients.         \\ \hline
		\textbf{CR.4}      &       Valid output clients for the system will be specified by the consumer.         \\ \hline
		\textbf{CR.5}      &       The hardware that the system runs on will support the runtime of our software solution.         \\ \hline
		\textbf{CR.6}      &       The system will have access to the network to which the data sources are connected.         \\ \hline
\end{longtable}

\subsubsection{Time}
\begin{longtable}[H]{| p{1.5cm} | p{15cm} |}
		\hline
		\cellcolor{titleColor}\textbf{ID}		&	\cellcolor{titleColor}\textbf{Description}	\\	\hline
		\textbf{CR.7}      &       The customer will receive an initial project plan detailing the project requirements no later than October 25th 2013.         \\ \hline
		\textbf{CR.8}      &       The customer will receive an interim report detailing project progress no later February 14th 2014.         \\ \hline
		\textbf{CR.9}      &       The customer will receive a final report detailing the proposed system no later than May 28th 2014.         \\ \hline
		\textbf{CR.10}      &       The customer will be presented with a system prototype on May 30th 2014.         \\ \hline
\end{longtable}

\subsubsection{Development}
\begin{longtable}[H]{| p{1.5cm} | p{15cm} |}
		\hline
		\cellcolor{titleColor}\textbf{ID}		&	\cellcolor{titleColor}\textbf{Description}	\\	\hline
		\textbf{CR.11}      &       The development team will be comprised of seven software engineers.         \\ \hline
		\textbf{CR.12}      &       Any requirement changes outside of this document will be negotiated separately.         \\ \hline
\end{longtable}


\section{Possible solutions}

When examining possible HUMS solutions, we inspected sensing, storing, analysing
and reporting data separately. This was done so that the best choice for each
could be individually selected, ensuring every sector of the system performs to
the highest possible standard.

When examining how the HUMS should sense new data we identified three distinct
options:
\begin{enumerate}
	\item Data input clients are directly linked to the system, with data being
	      pushed or polled depending on client support. This requires the system
	      to implement custom data specifications, which is likely to become
	      infeasible as the system expands across domains.
	\item Unmanaged data input clients which store data elsewhere. The HUMS polls
	      for data at the customers request. This option is more
	      inefficient than having data pushed to the system since polls may
	      frequently fail to collect data and therefore waste resources, or data
	      may otherwise be available but uncollected by the HUMS and thereby
	      leave it out-of-date. This approach, however, may integrate better with 
	      existing data input clients, reducing workload for the consumer.
	\item Provide an input API which data input clients wishing to push data to
	      the system must implement. This option is the most flexible, allowing
	      the system to be integrated with any consumer system, in any domain. It
	      does potentially mean more work for the consumer in migrating existing
	      systems to use the provided API. 
\end{enumerate} 
When considering options for how to store data there are two core methods, an
SQL or NoSQL database. Using an SQL database would provide the consumer with
guaranteed atomicity, consistency, isolation and durability (ACID), thereby
offering reliability over throughput. Migration for the consumer may be easier
when using an SQL database, as SQL has been popular among enterprises for a long
time. However, SQL does not easily scale horizontally, making it difficult to
expand one system across a large number of consumers. NoSQL solves the problem
of scalability at the expense of full ACID compliance. Polyglot persistence offers a
solution to all of the problems identified when choosing between SQL and NoSQL,
by allowing the use of both. When using polyglot persistence the type and
structure of every type of data must be examined to determine which datastore is
most suitable, whether SQL or NoSQL.

When designing the data analysis segment of the system there are three feasible
implementations:
\begin{enumerate}
	\item A plugin API, which allows users to implement their own analysis rules.
	      The system would provide multiple layers of abstraction, allowing the
	      consumer to take control of analysis at the level they see fit. For
	      example, using a low level API may be faster and allow greater
	      customisation, but a web API is more generic and easier for the consumer
	      to implement.
	\item Providing a rules engine for analysis, allowing the consumer to specify
	      and integrate their analysis requirements at a high level. This option
	      would be useful from the consumers perspective, abstracting from any low
	      level interfacing with the HUMS, but increasing the workload for our
	      team.
	\item Create customised single-purpose tools for each domain. This would be
	      massively inefficient from a development perspective, as well as costly
	      and inflexible. However the consumer may prefer to have a system
	      completely designed around their needs. 
\end{enumerate}
For the reporting segment of the HUMS there are again multiple options.
Consumers may wish to pull reports from the system at specified times, or want
the system to notify another system or end user when an event has occurred. When
considering the consumer requesting a data report, we could provide them with a
single output format such as PDF or XML, however this is not very flexible and
is unlikely to provide the required functionality for all consumers. Another
option is to allow the user to directly access the HUMS datastore, however this
would leave the system open to data inconsistencies and expose implementation
details to the consumer. A more flexible solution would be to provide a generic
reporting API to which plugins can be added. Each plugin would add a different
report format, allowing the consumer to choose which they wish to use in each
situation. This solution would be highly flexible, with new plugins being added
as the HUMS extends across domains.

When pushing notifications to another system or user, the HUMS could send events
encoded with a standard data interchange format, for example JSON or XML. This
would allow them to process the output themselves. The HUMS could also provide a
set of plugins allowing the consumer to choose what type of notification is
triggered when an event of a specific type occurs. For example, the system could
send an email for passive events but a mobile push notification or text message
for an event that requires urgent action. Allowing a variety of notification
channels provides a much more tailorable and maintainable product, but it
requires more development time and effort.

\subsection{Proposed solution}
Our proposed solution, shown in figure \ref{fig:systemDiagram}, uses the 
concept of a client-server architecture, with the consumer's system acting 
as the client and the server being the HUMS processing the data. 
The HUMS could exist within the consumer's data input client, or could be 
geographically-distant but connected on the same network or though the 
internet. The solution brings together the best elements from our
possible solutions for each section of the system, utilising a plugin
architecture for several components. 

The HUMS will provide an input API for data input clients to push new data to the
system, accepting any correctly formatted sensor data sent on the correct
network $(\textbf{DD.1})$. Achieving correctly formatted data may require changes
to the consumer's system, or use of middleware. This approach makes our system
flexible, supporting a wide range of data input clients, meaning in future the
solution could easily extend across domains. 

For data storage, the only applicable solution in this scenario is a database
$(\textbf{DD.2}).$ Using polyglot persistence gives us the option of choosing the
database technology most suited to the data, which we will determine
during development. To make our choice we will look at all the relevant
technologies available and evaluate strengths and weaknesses in relation to
other design decisions for the data being stored, and how it will be accessed. 

For analysis the system will provide generic plugins, as well as an API for
consumers to extend and create their own specialised analysis rules
 $(\textbf{DD.4})$. Offering an API allows the system to be tailored to different 
 consumers and domains, as generic analysis tools
could not hope to meet every consumer's needs by itself.  Domain-specific plugins 
will be provided to simplify the process for those performing common analysis
 functions. This might include graphical tools or other user-friendly features. 
 The range of plugins offered can be expanded as the system expands into 
 more domains. 

We will allow the consumer to customise the types of reports that can be 
created $(\textbf{DD.5})$. They will be able to request reports in a variety of 
formats, including human-friendly formats, such as PDF, and common data interchange
 formats, such as XML. Consumers can use these data formats as a data API to build their own
reports or for any other purpose. This will allow the consumer flexibility to produce 
what they want, whilst also offering them easy to use built-in tools.

Our system will provide plugins to support various types of notifications 
$(\textbf{DD.6})$ that can be triggered based upon customisable events. 
Notifications will include emails and mobile push, alongside more generic notifications formatted 
in standard formats such as XML. The generic notifications 
could then be processed further by the consumer, allowing them to tailor how 
notifications are used and delivered. 

An administration centre $(\textbf{DD.3})$ that allows
the consumer to configure any customisable system settings, including setting up
events that trigger notifications and defining data schemas, will also be included. This tool 
will be offered through an intuitive web based interface which is only accessible by the
authorised consumer. 

For the prototype we hope to demonstrate at the end of the project, we plan to
implement most of the solution described above. The core of the system,
including sensing, storing, analysing and reporting data, will be functional,
as will a small set of the domain specific components built utilising the plugin 
architecture. Some additional simple components may be required when 
demonstrating, such as a tool to simulate data being input to the system. 

\subsubsection{Design Decision}
\begin{table}[h!]
	\begin{tabular}{ | p{3cm} | p{3.5cm} | p{9.5cm} | }
		\hline
			\textbf{Design Decision ID}&	\textbf{Requirements Fulfilled}&	\textbf{Description}	\\ \hline
			\textbf{DD.1}&	FR.1, FR.2&	Provide an input API which data input clients wishing to push data to the system must conform to.	\\ \hline
			\textbf{DD.2}&	FR.7, FR.8&	Use a database to store data.	\\ \hline
			\textbf{DD.3}&	FR.4, FR.5, FR.6 FR.9, FR.10, FR.14, FR.15, FR.18, FR.19&	The system will provide an admin centre which allows the consumer to define data schemas and any configurable system settings.	\\ \hline
			\textbf{DD.4}&	FR.11&	The system will provide an analysis API, which the consumer can choose to implement, it will also provide a set of plugins conforming to this API which perform commonly used analysis functions.	\\ \hline
			\textbf{DD.5}&	FR.17&	The system will provide a set of plugins which allow the consumer to pull reports from the system in a variety of common formats, including high level formats such as PDF and low level formats such as XML.	\\ \hline
			\textbf{DD.6}&	FR.13, FR.13, FR.16&	The system will provide a set of plugins which push notifications of different formats to end users or data output clients.	\\ \hline
	\end{tabular}
\end{table}


\begin{figure}[H]
	\centering
	\includegraphics[width=0.65\textwidth]{system-architecture.pdf}
	\caption{System diagram}
	\label{fig:systemDiagram}
\end{figure}

\section{Team organisation}

\subsection{Team members}
A list of team members is provided below, accompanied by a list of roles and
assignments relevant for the course of the project. The responsibilities have
been divided in a such a way that effort for all team members is equal and
fairly distributed. Assignees have been given roles which, where possible,
utilise their identified skills, and minimise exposure to established
weaknesses.

\begin{tabular}{ p{4cm} p{4cm} p{4cm} p{4cm} }
	\textbf{AJF}-Adam Fahie & \textbf{AIF}-Andrew Fairbairn &
			\textbf{TF}-Anthony Free & \textbf{TD}-Tom Davies \\
	\textbf{JM}-Joseph Mansfield & \textbf{RT}-Rosy Tucker &
			\textbf{MW}-Michael Walker & \\
\end{tabular}
\subsection{Administrator Roles} 
\begin{table}[H]
\begin{tabular}{ | p{3.5cm} | p{8.5cm} | p{4.5cm} |}
\hline
\textbf{Role}	        &   \textbf{Description}                                                                   & \textbf{Assignees}                         \\ \hline
Point of Contact         &   Buffer between customer and team.                                        & RT                                                 \\ \hline
Team Coordinator      &   Manage team communication and organise meetings.            & JM           				         \\ \hline
Minutes Secretary     &   Take minutes during meetings.                                                & AJF,AIF,TF,TD,JM,RT,MW             \\ \hline

\end{tabular}
\end{table}

\subsection{Technical Roles}
\begin{longtable}[H]{ | p{3.5cm} | p{8.5cm} | p{4.5cm} |}
\hline
\cellcolor{titleColor}\textbf{Role}	       		&   \cellcolor{titleColor}\textbf{Description}                                                                   											& \cellcolor{titleColor}\textbf{Assignees}                  \\ \hline
Editors         			&   Review and format draft documentation.                                        											& JM,TD,RT                               \\ \hline
Software Architect      	&   Responsible for ensuring requirements are met, and software is consistent.         								& AJF         			 	\\ \hline
Requirements Analysis     &   Identification of customer needs and requirements.                                              								& RT,AJF,JM,TD,AIF 		\\ \hline
Developer         		&   Developing software to agreed standards.                                      		 								&  AJF,AIF,TF,TD,JM,RT,MW     \\ \hline
UX/UI Designer     		&   Accessibility, UI/UX and user stories.							                								& JM,RT,AIF           			 \\ \hline
Security     			&   Advising on relevant security standards and protocols.                                           								& TF,MW				   	 \\ \hline
Penetration Tester       	&   Test HUMS for exploits and security weaknesses.                           	                 							& AJF,TF,AIF,TD,MW              	\\ \hline
Software Tester     		&   Creating or assisting in the creation of tests.          													& AJF,AIF,TF,TD,JM,RT,MW  	 \\ \hline
Risk Manager     		&   Enhancing chance of a successful delivery.                                               									& RT.TD				   	 \\ \hline
Database Admin		&   Design of database strategies and schemas.                            		                 							& AJF,AIF,MW                           	 \\ \hline
Legal    				&   Avoiding copyright license and patent violations.          					       								& JM           				 \\ \hline
Quality Assurance             &   Ensure cohesion between system modules, analysing system performance and enforce code quality standards.  		& JM,TD				    	\\ \hline
Product Owner        		&   Representing the views of the customer.                                       										 	& RT                                        	\\ \hline
Scrum Master      		&   Ensuring Scrum process is followed by all team members, and ensuring good inter-team communication.            		& RT           				\\ \hline
\end{longtable}

\section{Development plan}

\subsection{Software engineering methodology}
When developing the HUMS system, we chose to take an agile approach to
development and team management. Our implementation will follow a plugin
architecture, with one central codebase providing the core functionality and
plugins which provide domain tailorability. If a non-agile approach was to be
taken, a plugin architecture would be infeasible as those methodologies require
all requirements and documentation to be completed before development commences.
For example, the V-model and Waterfall model are linear processes, where
requirements are only discussed and designed before implementation commences.
This linearity gives no support for changes in requirements during the later
phases. Given the time constraints and the need for the system to be able to
evolve into various domains, these linear development methodologies are
impractical.

Agile software development, however, allows the project to split into smaller
sub-projects called `iterations' \cite{hazzan2008agile}. Each iteration allows
for a different section of the project to be planned, documented, developed and
tested. It also allows for the development team to be split into smaller
sub-teams working simultaneously on different areas on the project. At the end
of each iteration project priorities can be re-evaluated and the team structure
altered. 

\cite{hazzan2008agile} identifies three key perspectives of software
development: human, organisational and technological. The scrum methodology
harnesses two of those perspectives, providing a platform for both team and
project organisation. Unfortunately scrum does not provide a clear approach to
development from a technological perspective, providing no clear guidelines on
how code should be created or structured. However, the test driven approach does
provide a clear method of developing software from a technological perspective.
This encourages the development of unit and acceptance tests before the
development of code, the code itself is then written to make the tests pass.
Using both methodologies side by side appears to be the best approach for this
project, allowing for a well managed and organised team to produce well
structured and reliable code. Feature driven development could be used in place
of test driven and would provide a good platform for plugin generation, allowing
each plugin to be designed and implemented independently. However, it conflicts
with the project organisation aspects of the scrum methodology, meaning they
could not be used together without creating confusion. Since the scrum method
provides more guidance for managing team structure and monitoring project
progress, feature driven development was discounted in favour of scrum.

\subsection{Schedule}
After examining the scope of the project we divided the first two phases into a schedule of tasks, with each task assigned to the team members responsible for the corresponding project role. We then organised the tasks into sprints, assigning work periods for each. Once organised the tasks were examined to ensure equal effort from all team members throughout the course of the project. Dependencies between tasks were then extracted, identifying a critical path through the project, ensuring, for every task, all prerequisite tasks are completed prior to the task commencing.  The flow of work can be seen in figure \ref{fig:Gantt} and shows an even work schedule throughout.

\begin{longtable}{ | l | l | l | l | l | l | }
\hline
ID & Task Name & Start & Finish & Dep. & Assignee \\ \hline \hline
1 & Initial problem analysis & 11/10 & 12/10 & & ALL \\ \hline
2 & Discussions with stakeholders & 12/10 & 17/10 & 1 & RT \\ \hline
3 & Requirements analysis & 14/10 & 18/10 & 1,2 & ALL \\ \hline
4 & Identify use cases & 19/10 & 20/10 & 3 & RT,TD \\ \hline
5 & Propose prototype solutions & 22/10 & 22/10 & 4 & AJF,AIF,TF, TD,JM,RT \\ \hline
6 & Decide on proposed solution & 22/10 & 23/10 & 5 & ALL \\ \hline
7 & Define team structure & 23/10 & 23/10 & & AJF,JM \\ \hline
8 & Identify risks & 22/10 & 23/10 & & TF \\ \hline
9 & Produce initial report & 19/10 & 23/10 & & ALL \\ \hline
\hline
10 & Further requirements engineering & 07/11 & 08/11 & & ALL \\ \hline
11 & Discussions with stakeholders & 07/11 & 09/11 & & RT \\ \hline
12 & Generate additional use cases & 09/11 & 09/11 & 10 & ALL \\ \hline
13 & Database software decision & 08/11 & 08/11 & & AJF,AIF \\ \hline
14 & Decision on tools and languages & 07/11 & 08/11 & & AJF \\ \hline
15 & Software development infrastructure & 09/11 & 10/11 & 14 & AJF,AIF,JM \\ \hline
\hline
16 & Sprint planning & 10/11 & 11/11 & 11-15 & ALL \\ \hline
17 & Software development & 11/11 & 25/11 & 16 & ALL \\ \hline
18 & Database development & 11/11 & 25/11 & 16 & AJF,AIF,MW \\ \hline
19 & Testing & 11/11 & 25/11 & 16 & AJF,TF,AIF, TD,MW \\ \hline
20 & Demo to customer & 25/11 & 26/11 & 17-19 & RT \\ \hline
21 & Review customer feedback & 25/11 & 26/11 & 17-19 & ALL \\ \hline
\hline
22 & Sprint planning & 26/11 & 27/11 & 21 & ALL \\ \hline
23 & Software development & 27/11 & 11/01 & 22 & ALL \\ \hline
24 & Database development & 27/11 & 11/01 & 22 & AJF,AIF,MW \\ \hline
25 & Testing & 27/11 & 11/01 & 22 & AJF,TF,AIF, TD,MW \\ \hline
26 & Demo to customer & 14/01 & 14/01 & 23-25 & RT \\ \hline
27 & Review customer feedback & 14/01 & 14/01 & 23-25 & ALL \\ \hline
\hline
28 & Sprint planning & 15/01 & 15/01 & 27 & ALL \\ \hline
29 & Software development & 16/01 & 29/01 & 28 & ALL \\ \hline
30 & Database development & 16/01 & 29/01 & 28 & AJF,AIF,MW \\ \hline
31 & Testing & 16/01 & 29/01 & 28 & AJF,TF,AIF, TD,MW \\ \hline
32 & Demo to customer & 29/01 & 30/01 & 29-31 & RT \\ \hline
33 & Review customer feedback & 29/01 & 30/01 & 29-31 & ALL \\ \hline
\hline
34 & Contingency & 31/01 & 14/02 & & \\ \hline
35 & Produce interim report & 10/11 & 14/02 & 12-15 & ALL \\ \hline
\end{longtable}

\begin{figure}[H]
	\centering
	\includegraphics[width =21cm, trim=4cm 2cm 0cm 1.5cm]{pdf-images/gant}
	\caption{A Gantt Chart showing the project flow of work} 
	\label{fig:Gantt}
\end{figure}


\section{Risk register}

A risk assessment has been performed as part of the initial report to ensure the
team is aware of any problems which could later arise, and to provide a guide as
to how to react when such problems occur. The hazards (risks) have been
identified and classified based upon team members' past experiences in similar
projects and group discussion. The areas these hazards impact were then
analysed, as well as the probability of occurrence. These are then weighted so
that we can identify the risks which are likely to have the greatest detrimental
effect on the project. The likelihood score (LS) and impact scores (IS) are
listed in the table below.

\begin{longtable}[H]{| p{0.6cm} | p{2cm} | p{0.3cm} | p{2.6cm} | p{8.1cm} | p{0.7cm} |}
    \hline
    \cellcolor{titleColor}\textbf{Risk ID}   & \cellcolor{titleColor}\textbf{Risk}                                             &\cellcolor{titleColor}\textbf{LS}        & \cellcolor{titleColor}\textbf{IS}                                 & \cellcolor{titleColor}\textbf{Mitigation and Contingency} & \cellcolor{titleColor}\textbf{RS} \\ \hline                                                                                                                                                                                                                                                                                                                                                                                                                                                                          
    $\textbf{R.1}$   & Short term loss of team members                  & $6$       & \textit{Moderate}
\newline Deadline failure                                        
      &  The team can then reactively reallocate the team member's work across remaining team members. To aid with this, the team must proactively ensure that no work relating to the project is outside of team version control. Use of the scrum methodology proactively aids work reallocation, ensuring team members are aware of all assigned work. 
      & $18$    \\ \hline
    $\textbf{R.2}$    & Long term loss of team members                   & $2$ & \textit{Catastrophic}
\newline Deadline failure and low standard of deliverables 
    & If a team member is unavailable for an extended period, the team will react by notifying the customer and possibly extending deadlines. The proactive procedures mentioned in \textbf{R.1} will also be followed to reduce the impact of this scenario.                                                                                                                                                                                                                                                                                            
    & $10$    \\ \hline
    $\textbf{R.3}$     & Short or long term loss of resources             & $2$ & \textit{Catastrophic}
\newline Deadline failure and loss of code base              
    & Proactive use of a source code repository, meaning code-base and history is decentralised. If the repository is lost, the data can be retrieved from the local repository copies and university backups.                                                                                                                                                                                                                                                                                                        
    & $10$    \\ \hline
    $\textbf{R.4}$     & Team member under-performance                    & $3$         & \textit{Major}
\newline Deadline failure and low standard  of deliverables            
    & Project plan must be feasible. The skills of the team as a whole, and individual team members must be proactively established early on and taken into account when assigning roles.                                                                                                                                                                                                                                                                                                                                                              
    & $12$    \\ \hline
    $\textbf{R.5}$    & Mis-interpretation of requirements                & $3$        & \textit{Major}
\newline Deliverables that are not valid                                        & 
    Requirements, design and implementation strategy must be proactively verified with the customer. This process is iterative, stopping when both customer and developers are content. Any changes to those requirements must result in re-negotiated deadlines.                                                                                                                                                                                                                                                                                    
    & $12$    \\ \hline
    $\textbf{R.6}$    & Slow response to customer queries                & $3$        & \textit{Major}
\newline Deadline failure and  low standard of deliverables         
    & Customer has assured a two working day response where possible. Further mitigation can be achieved by proactively communicating issues well in advance of deadlines.                                                                                                                                                                                                                                                                                                                                                                            
    & $12$    \\ \hline
    $ \textbf{R.7}  $   & Failure to produce required system functionality & $2$ & \textit{Catastrophic}
\newline Wasted time and loss of marks                        
    & Customer verification of requirements can counteract this risk. System testing, to ensure all agreed upon requirements are met, will also reduce this risk.                                                                                                                                                                                                                                                                                                                                                                                      
    & $10$    \\ \hline
   $ \textbf{R.8} $    & Missing internal team deadlines                  & $5$            & \textit{Moderate}
\newline Project falls behind due to missing dependencies                     
    & Perform critical path analysis to identify tasks which will take the longest time and which are a prerequisite to others. A greater team effort can then be assigned to these areas if it seems likely to miss a deadline or halt progress elsewhere.                                           
    &  $15$    \\ \hline
\end{longtable}


The likelihood score defines probability of something occurring. Utilising
\textit{Kents Words of Estimative Probability}\cite{kent1966strategic}, with
`certain' weighted $7$ and `impossible' weighted $1$.

\begin{table}[H]
	\begin{tabular}{  p{2cm}  p{1.5cm}  p{3cm} | p{2cm} | p{2cm} | p{2cm} | p{2cm} | p{2cm} | }
		\cline{4-8}
		& &	&		\multicolumn{5}{ |c| }{Impact Score (Least->Most)}&					\\ \cline{4-8}
		& &	&	1&	2&	3&	4&	5&	\\ \cline{4-8}
		& &	&	Negligible&	Minor&	Moderate&	Major&	Catastrophic&	\\ \hline
		\multicolumn{1}{ |c| }{\multirow{7}{*}{Likelihood Score}} &	7&	Certain&	7&	14&	21&	28&	35&	\\ \hline
		\multicolumn{1}{ |c  }{} &		6&	Almost certain&	6&	12&	18&	24&	30&	\\ \hline
		\multicolumn{1}{ |c  }{} &		5&	Probable&	5&	10&	15&	20&	25&	\\ \hline
		\multicolumn{1}{ |c  }{} &		4&	Chances about even&	4&	8&	12&	16&	20&	\\ \hline
		\multicolumn{1}{ |c  }{} &		3&	Probably Not&	3&	6&	9&	12&	15&	\\ \hline
		\multicolumn{1}{ |c  }{} &		2&	Almost certainly not&	2&	4&	6&	8&	10&	\\ \hline
		\multicolumn{1}{ |c  }{} &		1&	Impossible&	1&	2&	3&	4&	5&	\\ \hline
	\end{tabular}
\end{table}
														

\begin{table}[h!]
	\begin{tabular}{ | p{2cm} | p{4cm} | p{10cm} | }
		\hline
		\textbf{Score}&	\textbf{Risk Level}&	\textbf{Recommended Response}	\\ \hline
		\textbf{23-35}&	HIGH&	Mitigation plan is required. Immediate action is required.	\\ \hline
		\textbf{11-22}&	MEDIUM&	To be included in the action plan and reviewed.	\\ \hline
		\textbf{0-10}&	LOW&	Included in action plan in limited scope. Minimum review.	\\ \hline
	\end{tabular}
\end{table}


\section{Customer communication}

Knowing that the customer may not respond to emails rapidly, we began our
correspondence promptly after the project kick off. The customer did
respond quickly to all messages, however we feel that erring on the side of
caution and starting early was the best strategy. We followed up each response
with a team meeting within two days in order to discuss and implement any
changes or suggestions mentioned.

Communication so far has focussed on verifying that the outlined requirements
meet the expectations of the customer. We used the initial meeting $(9/10)$ to
clarify terminology and produce an initial set of requirements,  having
formatted these we sent them to the customer for feedback $(13/10)$.

Feedback $(14/10)$ suggested that our understanding of system tailorability was
inaccurate, with the customer providing clarification. After altering the
requirements in response and resending them to the customer $(16/10)$, we received
further feedback suggesting that the wording of our requirements was lacking in
parts.

We quickly remedied this before forwarding the changes to the customer $(19/10)$,
which resulted in helpful feedback allowing us to settle on the outlined list
of requirements. In producing and gradually refining the requirements, we began
to conceive an idea of how the system should function, and how it should be
tailorable. This has directly affected our proposed solution, by leading us
towards the outlined client-server architecture, where the HUMS acts as the
server. Having decided that we wanted to structure the HUMS as a framework with
support for plugins, we pitched this idea to the customer and received positive
and succinct feedback.

\section{Glossary}

\begin{description}

	\item[HUMS] Health and usage monitoring system(s).
	\item[Customer] Thales. 
	\item[Consumer] The recipient organisation of the system.
	\item[(End) User] An individual or organisation using the system.
	\item[Client] A piece of computer hardware or software which accesses a
	              service made available by the HUMS system.
	\item[The System] The HUMS we are designing and developing.
	\item[Event] A point of interest flagged up by an analysis system, which
	             may result in a notification.
	\item[Data input client] Anything that provides data to the HUMS through
	                         the input interface.
	\item[Data output client] Anything that receives data from the HUMS
	                          through the reporting or notification interfaces.
	\item[Notification] A message sent by the system as a result of an event
	                    being fired.
	\item[Report] A message sent by the system as a result of a request from a
	              user. 
	\item[HUMS Instance] One installation or occurrence of the system for a
	                     specific consumer.
\end{description}
\pagebreak
\bibliography{report-refs}
\bibliographystyle{IEEEtran}
\end{document}
