\documentclass[10pt,a4paper]{article}

\usepackage[margin=1in]{geometry}
\usepackage[UKenglish]{babel}
\usepackage{enumitem}
\usepackage{calc}
\usepackage{fancyhdr}
\usepackage{graphicx}
\usepackage{multirow}
\usepackage[table]{xcolor}
\usepackage{float}
\usepackage{longtable}
\usepackage{tabularx}
\usepackage{parskip}
\usepackage{soul}
\usepackage{ifthen}
\usepackage[compact]{titlesec}
\usepackage[justification=centering]{caption}
\usepackage{subcaption}
\usepackage{listings}
\usepackage{xcolor}

\input{listingsPreamble.tex}

\definecolor{reqColor}{RGB}{80,80,120}

%%Tables
\newcommand{\tableformat}[4]{
\begin{table}[ht!]
\centering
  \rowcolors{2}{gray!10} {white}
\def\arraystretch{1.5}
\begin{tabular}{#1}
  \hline
  \rowcolor[gray]{0.9} #2
  \hline
\end{tabular}
\caption{#3}
\label{#4}
\end{table}}

\newcommand{\xtableformat}[4]{
\begin{table}[ht!]
\centering
  \rowcolors{2}{gray!10} {white}
\begin{tabularx}{\textwidth}{#1}
  \hline
  \rowcolor[gray]{0.9} #2
  \hline
\end{tabularx}
\caption{#3}
\label{#4}
\end{table}}

\pagestyle{fancy}
\lhead{T Davies, A Fahie, A Fairbairn, A Free, J Mansfield, R Tucker, M 
Walker}
\chead{}
\rhead{GPIG-C}
\cfoot{\vspace{-0.6cm} \thepage}

\setlist{nolistsep} % Reduces lots of white space around lists

\renewcommand{\headrulewidth}{0.4pt} % Add rules below header
\renewcommand*{\thefootnote}{\fnsymbol{footnote}}

\newcommand{\conreq}[1]{\textcolor{reqColor}{\textbf{CR.#1}}}
\newcommand{\fr}[1]{\textcolor{reqColor}{\textbf{FR.#1}}}
\newcommand{\ed}[1]{\textcolor{reqColor}{\textbf{ED.#1}}}
\newcommand{\nfr}[1]{\textcolor{reqColor}{\textbf{NFR.#1}}}
\newcommand{\qas}[1]{\textcolor{reqColor}{\textbf{QAS.#1}}}

%%Scenarios
\newenvironment{scenario}[1]{
\newcommand{\source}[1]{\item[Source of Stimulus:] ##1}
\newcommand{\stimulus}[1]{\item[Stimulus:] ##1}
\newcommand{\artifact}[1]{\item[Artifact:] ##1}
\newcommand{\environment}[1]{\item[Environment:] ##1}
\newcommand{\response}[1]{\item[Response:] ##1}
\newcommand{\measure}[1]{\item[Response Measure:] ##1}
\newcommand{\rationale}[1]{\item[Scenario Rationale:] ##1}
\newcommand{\quality}[1]{\item[Quality:] ##1}
		\begin{description} [noitemsep]	
		\item[Scenario ID:] \qas{#1}
		}{\end{description} \vspace*{0.3cm}
		}

%%Requirements
\newenvironment{requirements}{
\newcommand{\requirement}[4]{\item[##1{##2}] ##3
							\ifx&##4&
							%nothing
							\else
								\begin{description}
									##4
								\end{description}							
							\fi
							}
		\begin{description}[noitemsep, leftmargin=1.3cm]	
		}{\end{description} \vspace*{0.3cm}
		}
		
\begin{document}
\begin{center}
{\vspace*{-0.5cm}
\Huge GPIG-C Final Report}
\vspace*{0.2cm}

\vspace*{0.1cm}

Wednesday, 21st May 2014
\end{center}
\vspace*{0.4cm}
\hrule
\vspace*{0.4cm}

%-------------------------------------------------------------%
%----------------------LAUNCHING -------------------%
%-------------------------------------------------------------%
\section{Launching the HUMS}
\label{sec:launching}
\subsection{Software required to use the core}
To utilise the HUMS, the target system must have Java 7 installed. This can be obtained for free for Windows, Linux, and Mac from www.java.com.

\subsection{Launching the core}
\subsubsection{Windows}
To launch the core on Windows, navigate to the directory in windows explorer where core.jar is located. The core can be launched by double clicking on the Core.jar file. This will launch the HUMS.
\subsubsection{Linux}
To launch the core on linux, using the terminal, navigate to the directory where Core.jar is located. The HUMS can then be launched utilising the command:
\begin{center}
\textit{./Core.jar}
\end{center}
Alternatively, it can be opened by navigating to the directory containing Core.jar in a file browser. Right click on the file browser and select the option "Run with Java" or "Run with OpenJDK".

\subsubsection{MAC}
To launch the core on Mac, using the terminal, navigate to the directory where core.jar is located. The HUMS can then be launched utilising the command:
\begin{center}
\textit{java -XstartOnFirstThread -jar Core.jar}
\end{center}

\section{Configuration}
\label{sec:setupcore}
\subsection{Loading a configuration file}


\end{document}
